%%%%%%%%%%%%%%%%%%%% Computational Framework %%%%%%%%%%%%%%%%%%%%

\section{Computational Framework for Monadic AI}
\label{section:computational}
Our approach slices the monads of the existing approach into reusable,
language-agnostic monad transformer stacks.
%
In Section ? we will establish exactly the space of monad transformers which,
given specific properties can be proven about them, will integrate seamlessly
within our framework.  For now we introduce a few familiar monad transformers
for the sake of demonstration, and defer the explanation of \textit{why} they
can be used.

\begin{minipage}{\linewidth}
\begin{lstlisting}
Identity     A := A
StateT   S m A := S $\to$ m (S $\times$ A)
State    S   A := StateT Identity A
Set          A := {..., A, ...}
SetT       m A := m (Set A)
\end{lstlisting}
\end{minipage}
Next, we introduce interfaces for various monadic behaviors and their laws:
\begin{minipage}{\linewidth}
\begin{lstlisting}
Monad (m : Type -> Type) :=
  return : $\forall$ A, A $\to$ m A
  bind : $\forall$ A B, m A $\to$ (A $\to$ m B) $\to$ m B
  left_unit : bind aM ret = aM
  right_unit : bind (ret x) f = f x
  associativity : bind (bind aM f) g = bind aM ($\lambda$ x $\to$ bind (f x) g)
MonadFunctor (t : (Type -> Type) -> Type -> Type):= 
  mapMonad : $\forall$ m, Monad m -> Monad (t m)
MonadPlus m | Monad m :=
  mzero : $\forall$ A, m A 
  _<+>_ : $\forall$ A, m A -> m A -> m A
  unit : aM <+> mzero = aM
  left_zero : bind mzero f = mzero
  right_zero : bind aM ($\lambda$ x $\to$ mzero) = mzero
  associativity : (aM <+> bM) <+> cM = aM <+> (bM <+> cM)
  commutativity : aM <+> bM = bM + aM
MonadPlusFunctor t | MonadFunctor t :=
  mapMonadPlus : $\forall$ S m, MonadPlus m -> MonadPlus (t m)
MonadState (S : Type) m | Monad m :=
  getState : m S
  putState : S $\to$ m unit
MonadStateFunctor t | MonadFunctor t := 
  mapMonadState : $\forall$ S m, MonadState S m -> MonadState S (t m)
\end{lstlisting}
and prove the following lemmas:
\begin{lstlisting}
forall S:
  - MonadFunctor (StateT S)
  - MonadPlusFunctor (StateT S)
  - MonadStateFunctor (StateT S)
forall S and m | Monad m:
  - MonadState S (StateT S m)

MonadFunctor SetT
MonadStateFunctor SetT
$\forall$ m | Monad m:
  - MonadPlus (SetT m)
\end{lstlisting}
\end{minipage}

We define \textbf{mjoin} for transporting values from the Set monad to an
arbitrary MonadPlus:
\begin{lstlisting}
mjoin : forall A m | MonadPlus m, Set A -> m A
mjoin {} = mzero
mjoin {..., x, ...} = ... <+> return x <+> ...
\end{lstlisting}

\subsection{Methodology}
Languages are embedded in the framework using the following methodology:
\begin{itemizenobreak}
  \item 
  Define a \textit{single} language state space, to be used for both concrete
  and abstract interpreters.
  \item 
  Define a \textit{single} \textit{monadic} small-step function, parameterized
  by a monadic effect interface and $\delta$ function.
\end{itemizenobreak}
For example, consider the following state space for a very simple arithmetic
language:
\begin{lstlisting}
Int := { ... the integers ... }
Symbol := { ... x, y, etc ... }
Sign := IsZero | IsPos | IsNeg
AInt := Exact Int | Symbolic Sign
Lang := Halt Atomic 
      | LetInc Symbol Atomic Lang 
      | IfZero Atomic Lang Lang 
Atomic := Num AInt | Var Symbol
Env A B := A -> Option B
\end{lstlisting}
Notice that \textbf{Lang} is a \textit{mixed} concrete-abstract language,
potentially capable of expressing both concrete and abstract semantics.
%
In this example we will call the delta function \textbf{inc}, because it
interprets increment for atomic expressions. Given \textbf{$\langle$m,
inc$\rangle$} where:
\begin{lstlisting}
- Monad m
- MonadPlus m
- HasState (Env Symbol (Set AInt)) m
- inc : AInt $\to$ AInt $\to$ Set Ant
\end{lstlisting}
The semantics for the language \textbf{AInt} can be defined:
\begin{lstlisting}
lookup :: Symbol -> Env Symbol (dom AInt) -> m (dom AInt)
lookup x e = case e x of
  None -> mzero
  Some aD -> return aD

atomic : Atomic -> m (dom AInt)
atomic (Num n) = return (return n)
atomic (Var x) = do
  e <- getEnv
  v <- lookup x e
  return v

step : Lang -> m Lang
step (Halt a) = return (Halt a)
step (LetInc x na l) = do
  n <- mjoin (atomic na)
  r <- delta n
  modifyEnv (insert x r)
  return l
step (IfZero ca tb fb) = do
  c <- mjoin (atomic ca)
  case c of
    Exact 0 -> return tb
    Exact _ -> return fb
    Symbolic IsZero -> return tb
    Symbolic IsNeg -> return fb
    Symbolic IsPos -> return fb
\end{lstlisting}

This single semantic step function is capable of expressing both the concrete
and abstract semantics of our arithmetic language.
%
To relate our monadic function back to small-step abstract state machines we
introduce another interface which all monads in the framework must obey:
%
\begin{lstlisting}
MonadStateSpace m :=
  ss : Type -> Type
  transition : $\forall$ A B, (A -> m B) -> ss A -> ss B
MonadStateSpaceFunctor t :=
  mapMonadStateSpace : $\forall$ m, MonadStateSpace m -> MonadStateSpace (t m)
\end{lstlisting}
%
and we prove the following lemmas:
\begin{lstlisting}
$\forall$ S, MonadStateSpaceFunctor (StateT S) where 
  ss (StateT S m) A := $\text{ss}_m$ (A $\times$ S)

MonadStateSpaceFunctor SetT where
  ss (SetT m) A := $\text{ss}_m$ (Set A)
\end{lstlisting}

[transition] establishes a relationship between the space of actions in the
monad and the (pure) space of transitions on an abstract machine state space.
%
If \textbf{m} is shown to implement MonadStateSpace, then the semantics of our
language can be described as the least fixed point of iterating [transition
step] from e, i.e.:  $\llbracket e \rrbracket = \mu\; x \to e \sqcup
\text{transition}\; x$

To recover the various semantics we are interested in, the monad \textbf{m} can
then be instantiated with following monad transformer stacks:
\begin{lstlisting}
  m1     := EnvStateT (Env Symbol (Set AInt)) Set
  ss m1 A $\approx$ Set (A $\times$ Env Symbol (Set AInt))
  m2     := SetT (EnvState (Env Symbol (Set AInt)))
  ss m2 A $\approx$ Set A $\times$ Env Symbol (Set AInt)
\end{lstlisting}
and inc functions:
\begin{lstlisting}
  inc-concrete : AInt -> Set AInt
  inc-concrete (Exact n) = return (Exact (n + 1))
  inc-concrete (Symbolic IsZero) = return (Symbolic IsPos)
  inc-concrete (Symbolic IsNeg) = return (Symbolic IsNeg) 
                              <+> return (Symbolic IsZero)
  inc-concrete (Symbolic IsPos) = return (Symbolic IsPos)

  inc-abstract : AInt -> AInt -> Set AInt
  inc-abstract (Exact n) | n < -1  = return IsNeg
                         | n == -1 = return IsZero
                         | n > -1  = return IsPos
  inc-abstract (Symbolic IsZero) = return (Symbolic IsPos)
  inc-abstract (Symbolic IsNeg) = return (Symbolic IsNeg) 
                              <+> return (Symbolic IsZero)
  inc-abstract (Symbolic IsPos) = return (Symbolic IsPos)
\end{lstlisting}

When instantiated with \textbf{$\langle$m1, inc-concrete$\rangle$} a
\textit{concrete} interpreter is recovered.  When instantiated with
\textbf{$\langle$m1, inc-abstract$\rangle$} a computable \textit{abstract}
interpreter is recovered.  When instantiated with \textbf{$\langle$m2,
inc-abstract$\rangle$} a computable \textit{abstract} interpreter with
\textit{heap widening} is recovered.


