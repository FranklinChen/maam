%%%%%%%%%%%%%%%%%%%% Introduction %%%%%%%%%%%%%%%%%%%%

To demonstrate our monadic abstract interpretation framework we guide the
reader through the design and implementation of 0CFA for a CPS language with
small-step semantics.
%
We present the process guided by three approaches to abstract interpretation:
%
\begin{itemizenobreak}
  \item Ad hoc design and ``a posteriori'' verification
  \item Abstracting Abstract Machines a la Van Horn and Might~\cite{van-horn:2010:aam}
  \item Using our monadic framework
\end{itemizenobreak}
%
General computational, correctness and modularity properties of our framework
are deferred to sections~\ref{section:computational},~\ref{section:correctness}
and~\ref{section:modularity}.
%
0CFA is chosen because it is well-studied, challenging, and useful for analysis of
higher order languages.
%
CPS is chosen to simplify control flow, and also for being well-studied in the
context of 0CFA.

%-----%

Our framework, and the techniques presented in this paper, are specialized to
the traditions of abstract interpretation pioneered by Cousot and Cousot
\cite{cousot:1977:ai}, and small-step operational semantics pioneered by
Friedman and Felleisen \cite{felleisen:1986:cek}.
%
Consequently, each semantics presented will take the form of a state space
$\DSS$ and transition function $\Dstep \in \DSS \to \DSS$, and each static
analysis will take the form of an abstract state space $\DSSAbs$ and abstract
transition function $\DstepAbs \in \DSSAbs \to \DSSAbs$.
%
Possibilties of generalizing to other approaches for semantics are discussed in
section~\ref{section:future-work}.
%
We will use Haskell as both an implementation language and semantics
meta-language.

%%%%%%%%%%%%%%%%%%%% Concrete Semantics %%%%%%%%%%%%%%%%%%%%

\subsection{Concrete Semantics}

Before discussing various approaches to the design and implementation of 0CFA
for CPS, we first present the semantics of CPS using Haskell.

%-----%


%%%%%%%%%%%%%%%%%%%% Ad-hoc Design %%%%%%%%%%%%%%%%%%%%

\subsection{Ad-hoc Design and ``A Posteriori'' Verification }

In the ``a posteriori'' approach to abstract interpretation the analysis
designer invents:
%
\begin{itemizenobreak}
\item (1.a) Abstract state space $\DSSAbs$
\item (1.b) Abstract step function $\DstepAbs \in \DSSAbs \to \DSSAbs$
\end{itemizenobreak}
%
and then attempts to establish correctness properties in an ad-hoc fashion.
%
In the abstract interpretation framework, soundness and tightness are
demonstrated through:
%
\begin{itemizenobreak}
\item (2.a) A Galois connection $\DSS \galois{\alpha}{\gamma} \DSSAbs$
\item (2.b) A proof that 
      $\alpha(\Dstep) \sqsubseteq \DstepAbs$
      or
      $\Dstep \sqsubseteq \gamma(\DstepAbs)$
      (they are equivalent)
\end{itemizenobreak}
%
These points are depicted in the following diagram:
%
\begin{center}
\begin{tabular*}{0.66\textwidth}{@{\extracolsep{\fill}} c c}
  \begin{tikzpicture}[ampersand replacement=\&]
    \matrix (m) [matrix of math nodes,row sep=3em,column sep=4em,minimum width=2em]
    { \DSS    \& \DSS    \\
      \DSSAbs \& \DSSAbs \\
    } ;
    \path[-stealth]
    (m-1-1) edge [bend right=40] node [left] {$\alpha$} 
                                 node [right] {\tiny(2.a)} (m-2-1)
    (m-2-1) edge [bend right=40] node [right] {$\gamma$} (m-1-1)

    (m-1-2) edge [bend right=40] node [left] {$\alpha$} (m-2-2)
    (m-2-2) edge [bend right=40] node [right] {$\gamma$} (m-1-2) 

    (m-1-1) edge node [above] {$\Dstep$} (m-1-2)
    (m-2-1) edge node [above] {$\DstepAbs$} 
                 node [below] {\tiny(1.b)} (m-2-2)
    ;
    \draw 
      (m-2-1) node [left=0.25cm] {\tiny(1.a)} 
    ;
  \end{tikzpicture}
  &
  \begin{tikzpicture}
    \matrix (m) [matrix of math nodes,row sep=3em,column sep=4em,minimum width=2em]
    { \DstepAbs \\
      \Dstep    \\
    } ;
    \path
    (m-1-1) edge node [right] {$\sqsubseteq$} 
                 node [left] {\tiny(2.b)} (m-2-1)
    ;
  \end{tikzpicture}
\end{tabular*}
\end{center}


%-----%

In the ad-hoc approach to static analysis, none of these steps are automated,
nor does the designer follow a methodology for how to proceed.
%
However this gives the analysis designer ultimate freedom and is simultaneously
the appeal of the approach, lending it to be used most often in practice.

%-----%


We present what the analysis designer might come up with for (1.a) and
(1.b) in designing 0CFA, and defer the ``a posteriori'' proofs of (2.a) and
(2.b) to prior work \cite{shivers:1991:cfa, nielson:2004:program-analysis,
van-horn:2010:aam, midtgaard:2008:calculational-cfa}.

%-----%

The state space for 0CFA, and all abstract state spaces to follow,
differ only in the \lstinline|Val|, \lstinline|Env| and $\DSS$ types.
%
\begin{donotbreak}
\lstinputlisting{hs/SSAbstract.hs}
\end{donotbreak}
%
A single global environment \lstinline|Env| is used to approximate all
environments in the concrete image of the abstract state space, which
corrosponds to heap widening in the literature\cite{shivers:1991:cfa}.
%
To maintain soundness in the face of a single global environment, we introduce
a \lstinline|Set| of values in the image of the environment, as it may need to
approximate the binding bindings with all possible 
%
The state space $\DSS$ contains a set of \lstinline|Call| states
because the abstract semantics are non-deterministic.
%
As a consequence of the global environment abstraction, closures no longer
need to capture their environment.
%
This cuts the recursion in the state space allowing $\DSS$ to be finite and
therefore the analysis to be computable.

%-----%

The abstract step function $\DstepAbs$ exhibits a few key differences from the
concrete step function $\Dstep$.
%
\lstinputlisting{hs/SSAbstract.hs}
%
Interpretations of literal constants and primitive operations truncate to a
symbolic type tag.
%
Non-determinism is introduces in both the denotation for \lstinline|Atomic| and
operational behavior of \lstinline|Call| when multiple values are returned from
the environment.
%
The global environment is computed as the join of all environments reached
after a single step, soundly abstracting each individual reachable environment.

%-----%

In summary, we claim advantages and deficiencies of the ad-hoc ``a posteriori''
approach are:
\begin{itemizenobreak}
\item + No restrictive framework in which to funnel the desired analysis
\item + Most ``off the shelf'' analyses from the literature use this method
\item - The analysis chosen is an arbitrary point on a spectrum of possible design choices
\item - Non-modular: Altering the analysis requires reconstructing all points (1.a) - (2.b)
\end{itemizenobreak}

%%%%%%%%%%%%%%%%%%%% AAM %%%%%%%%%%%%%%%%%%%%

\subsection{Abstracting Abstract Machines}

In the abstracting abstract machines approach to abstract interpretation, the
analysis designer:
%
\begin{itemizenobreak}
\item (1.a) Is guided to a particular abstract state space $\DSSAbs$ by a systematic methodology
\item (1.b) Invents abstract step function $\DstepAbs \in \DSS \to \DSS$
\end{itemizenobreak}
%
and in making the correctness argument must establish: 
%
\begin{itemizenobreak}
\item (2.a) A Galois connection $\DSS \galois{\alpha}{\gamma} \DSSAbs$ is
      derived alongside (1.a) by the systematic methodology
\item (2.b) The proof that $\alpha(\Dstep) \sqsubseteq \DstepAbs$ in an ad-hoc
      manner
\end{itemizenobreak}
%
These points are depicted in the following diagram, where grayed out names and
arrows indicate getting something for free (literally or methodologically):
%
\begin{center}
\begin{tabular*}{0.66\textwidth}{@{\extracolsep{\fill}} c c}
  \begin{tikzpicture}[ampersand replacement=\&]
    \matrix (m) [matrix of math nodes,row sep=3em,column sep=4em,minimum width=2em]
    { \DSS                                 \& \DSS                                 \\
      \node [lightgray] (m-2-1) {\DSSAbs}; \& \node [lightgray] (m-2-2) {\DSSAbs}; \\
    } ;
    \path[-stealth]
    (m-1-1) edge [lightgray,bend right=40] node [lightgray,left] {$\alpha$} 
                                           node [lightgray,right] {\tiny(2.a)} (m-2-1)
    (m-2-1) edge [lightgray,bend right=40] node [lightgray,right] {$\gamma$} (m-1-1)

    (m-1-2) edge [lightgray,bend right=40] node [lightgray,left] {$\alpha$} (m-2-2)
    (m-2-2) edge [lightgray,bend right=40] node [lightgray,right] {$\gamma$} (m-1-2) 

    (m-1-1) edge node [above] {$\Dstep$} (m-1-2)
    (m-2-1) edge node [above] {$\DstepAbs$} 
                 node [below] {\tiny(1.b)} (m-2-2)
    ;
    \draw 
      (m-2-1) node [lightgray,left=0.25cm] {\tiny(1.a)} 
    ;
  \end{tikzpicture}
  &
  \begin{tikzpicture}
    \matrix (m) [matrix of math nodes,row sep=3em,column sep=4em,minimum width=2em]
    { \DstepAbs \\
      \Dstep    \\
    } ;
    \path
    (m-1-1) edge node [right] {$\sqsubseteq$} 
                 node [left] {\tiny(2.b)} (m-2-1)
    ;
  \end{tikzpicture}
\end{tabular*}
\end{center}


%-----%

The derivation process for (1.a) and (2.a) in the Abstracting Abstract Machines
framework begins with identifying recursion in the state space \lstinline|SS|,
and cutting it with a level of indirection.
%
\begin{lstlisting}
data Val = LitV Lit 
         | CloV [Name] Call Env
type Env = Map Name Val                  
type SS  = Maybe (Call, Env)
\end{lstlisting}
%
becomes
%
\begin{lstlisting}
data Val   = LitV Lit
           | CloV [Name] Call Env
type Env   = Map Name Integer
type Store = Map Integer Val
type SS    = Maybe (Call, Env, Store)
\end{lstlisting}
%
A semantics for the new state space allocates unique addresses in the global
store when binding in a local environment.
%
This new semantics can be shown to be equivalent to the initial semantics,
allowing us to recover the concrete semantics with our new ``instrumented''
semantics.

%-----%


To arrive at a static analaysis, the designer must continue to cut infinite
portions of the state space.
%
The Abstracting Abstract Machines framework guided us to trade recursion for
indirection through the integers, and now we need only abstract the integers to
something finite to achieve a finite state space.
%
\begin{lstlisting}
data Val   = LitV Lit
           | CloV [Name] Call Env
type Env   = Map Name Integer
type Store = Map Integer Val
type SS    = (Maybe (Call, Env), Store)
\end{lstlisting}
%
becomes
%
\begin{lstlisting}
data Val   = LitV Lit
           | CloV [Name] Call Env
type Env   = Map Name Name
type Store = Map Name Val
type SS    = (Maybe (Call, Env), Store)
\end{lstlisting}
%
We then notice that \lstinline|Env| will always be the identity map, and
eliminate it from closures and the state space.
%
\begin{lstlisting}
data Val   = LitV Lit
           | CloV [Name] Call
type Store = Map Name Val
type SS    = (Maybe Call, Store)
\end{lstlisting}
%
Finally, we notice that during execution of the semantics, we necessarily must
reuse abstract addresses, which are now \lstinline|Name|s.
%
Because we must reuse addresses, we must do something sensible (meaning sound)
when two values are bound to the same address.
%
To solve this we introduce non-determinism into the semantics, lifting abstract
values to a powerset and taking the join of abstract values as they collide in
the store.
%
\begin{lstlisting}
data Val   = LitV Lit
           | CloV [Name] Call
type Store = Map Name (Set Val)
type SS    = (Set Call, Store)
\end{lstlisting}

%-----%

In the ad-hoc design, we called the global environment \lstinline|Env|, where
here we use \lstinline|Store|.
%
This is an ontological artifact of the systematic design process of Abstracting
Abstract Machines, where the store was introduced purely to cut recursion in
the state space.

%-----%

The above derivation of abstract state space is also amenable to derivation of
Galois connection.
%
The first step, which introduces concrete addresses with integers, is
isomorphic to the inital state space (under assumptions of heap
well-formedness, which can trivially be shown is preserved by the semantics).
%
Abstracting concrete addresses (integers) with abstract addresses (names) forms
a galois connection between state spaces.
%
This galois connection is difficult to fully justify (Van Horn and Might sketch
the proof; see \cite{midtgaard:2008:calculational-cfa} for all the details),
but can be done once and for all within the methodolgy, and applied freely by
the analysis designer.

%-----%

The benefits of the Abstracting Abstract Machines framework extend beyond a
methodology for designing a \textit{single} abstract interpreter.
%
There is a pivot point intentionally placed in the design process--between
introducing addresses and abstracting addresses--where the designer is set up
to recover a wide range of possible analyses.
%
The idea is then to implement an interpreter once parameterized by the choice
of abstract address (and abstract time, to recover the full AAM framework),
which is capable of recovering a wide range of analyses.
%
This is benefical to the designer as properties of the analysis change, or if
multiple analyses of varying precision and power are desired.

%-----%

However, certain classs of design decisions in constructing analyses are
\textit{not} captured by the pivot point in AAM.
%
For example: heap widening, abstract garbage collection and shape analysis each
cannot be recovered by a single abstract interpreter parameterized by abstract
address and time.
%
While Van Horn and Might have demonstrated the seamless integration of these
ideas into their framework, they do not come for free, and must still be
integrated in an ad-hoc manner for the analysis designer.

%-----%

In summary, we claim advantages and deficiencies of the Abstracting Abstract
Machines approach are:
\begin{itemizenobreak}
\item + Nearly systematic design of abstract state space $\DSSAbs$
\item + Pivot point in analysis design that enables a single parameterized implementation
\item - Constructing abstract step function $\DstepAbs$ is non-automated
\item - Pivot point in analysis design doesn't capture certain important
        classes of analysis design decisions
\end{itemizenobreak}

%%%%%%%%%%%%%%%%%%%% Calculation %%%%%%%%%%%%%%%%%%%%

\subsection{Calculation}

In the calculational approach to abstract interpretation, the analysis designer:
%
\begin{itemizenobreak}
\item (1.a \& 2.a) Constructs a Galois connection $\DSS \galois{\alpha}{\gamma}
                   \DSSAbs$ as the composition of several smaller well known
                   components, inducing the abstract state space $\DSSAbs$ .
\item (1.b \& 2.b) Performs calculation on the induced abstract semantics
                   $\DstepAbs = \gamma \comp \Dstep \comp \alpha$ to arrive at
                   a computable abstract semantics.
\end{itemizenobreak}
%
These points are depicted in the following diagram, where grayed out names and
arrows indicate getting something for free (literally or methodologically):
%
\begin{center}
\begin{tabular*}{0.66\textwidth}{@{\extracolsep{\fill}} c c}
  \begin{tikzpicture}[ampersand replacement=\&]
    \matrix (m) [matrix of math nodes,row sep=3em,column sep=4em,minimum width=2em]
    { \DSS                                 \& \DSS                                 \\
      \node [lightgray] (m-2-1) {\DSSAbs}; \& \node [lightgray] (m-2-2) {\DSSAbs}; \\
    } ;
    \path[-stealth]
    (m-1-1) edge [bend right=40] node [left] {$\alpha$} 
                                 node [right] {\tiny(2.a)} (m-2-1)
    (m-2-1) edge [bend right=40] node [right] {$\gamma$} (m-1-1)

    (m-1-2) edge [bend right=40] node [left] {$\alpha$} (m-2-2)
    (m-2-2) edge [bend right=40] node [right] {$\gamma$} (m-1-2) 

    (m-1-1) edge node [above] {$\Dstep$} (m-1-2)
    (m-2-1) edge [lightgray] node [lightgray,above] {$\DstepAbs$} 
                             node [lightgray,below] {\tiny(1.b)} (m-2-2)
    ;
    \draw 
      (m-2-1) node [lightgray,left=0.25cm] {\tiny(1.a)} 
    ;
  \end{tikzpicture}
  &
  \begin{tikzpicture}
    \matrix (m) [matrix of math nodes,row sep=3em,column sep=4em,minimum width=2em]
    { \node [lightgray] (m-1-1) {\DstepAbs}; \\
      \Dstep    \\
    } ;
    \path
    (m-1-1) edge [lightgray] node [right] {$\sqsubseteq$} 
                             node [left] {\tiny(2.b)} (m-2-1)
    ;
  \end{tikzpicture}
\end{tabular*}
\end{center}


%-----%

The components used to construct (1.a \& 2.a) are typicall typically drawn from
a catalogue of known techniques such as independent attribute method, direct
product, reduced product, total function space, relational method, direct
tensor product, reduced tensor product and monotone function space, to name a
few~\cite{nielson:2004:program-analysis}.
%
After constructing a final Galois connection, the abstract semantics are
derived through a process of calculation--a mixture of equational and partial
order reasoning--over the induced abstract step function $\DstepAbs = \gamma
\comp \Dstep \comp \alpha$.


%%%%%%%%%%%%%%%%%%%% Monadic Derivation %%%%%%%%%%%%%%%%%%%%

\subsection{Monadic Derivation}

Monadic abstract interpretation was introduced by Sergey et al. as a method for
simultaneously defining multiple versions of a semantics in one go.
%
The primary contribution of this work is a proof framework to augment this
approach, showing that the derived abstract interpreters are correct by
construction.
%
For now we contrast the burdon of proof for the analysis designer with the a
posteriori verification described in the previous section.

First we translate the semantics of CPS into monadic form as shown in figure
?, being careful to leave the semantics general enough to capture both concrete
and abstract semantics.
%
We note that certain design decisions must be made up front, like what the
abstract value space looks like, but defer the full choice of state space to a
particular instantiation within the monadic framework
%
Independent of other languages, we need only justify that mstep is monotonic,
which is true by simple inspection.


We now recover both the concrete semantics and multiple abstract semantics from
the monadic semantics by:
%
\begin{itemizenobreak}
  \item Designing delta and literal semantic functions
  \item Building an instance of the abstract monadic effect interface
\end{itemizenobreak}
