% This approach is representative of how (sound) static analyses are generally
% justified in practice.
% 
% The abstract state space and step function for 0CFA are shown in
% figure~\ref{cps:abstract}.
% %
% Several seemingly arbitrary choices were made in the design of the abstract
% state space shown.
% %
% Once this choice is made and the Galois connection relating it to the concrete
% state space is established, the abstract step function, and therefore the
% entire static analysis, is uniquely uniquely determined.
% %
% Exploiting this uniqueness to \textit{derive} rather than \textit{verify} the
% abstract step relation underlies the calculational approach, which is
% demonstrated in the next section.
% %
% For now, demonstrating a Galois connection (Maybe (Call, Env))
% \galois{\alpha}{\gamma} (Set Call, AEnv), showing that \textbf{astep} is
% monotonic, and showing that $(\alpha(\text{step}) \sqsubseteq
% \widehat{\text{step}})$ are all proof burdens for the analysis designer.
% %
% We do not detail such proofs here, rather we refer the reader to [Shivers] and
% [Might].
% 
% We make three observations in the relationship between the concrete and
% abstract semantics thus far.
% \begin{itemizenobreak}
% \item
% The choice of abstract state space was in some sense arbitrary.
% %
% For example, we could have chosen (Set (Call, Env)), yielding a flow-sensitive
% analysis.
% %
% \item
% \textbf{acall} would look much prettier if written in monadic style.
% %
% This observation was the initial inspiration for monadic abstract
% interpretation; we'll see how that story plays out in a later section.
% %
% \item
% Modifications to either the concrete or abstract semantics will likely render
% the other incompatible.
% \end{itemizenobreak}
% %
% We argue this last point is the primary point of pain in designing abstract
% interpreters.
% %
% For a large real-world language, the semantics may subtly change during
% development.  Even more likely, the analysis will be augmented and improved
% over time, as new questions wish to be asked of the analysis.
% %
% Maintaining each of the semantics (and their proofs) to agree on modifications
% is time consuming and error prone.
% %
% It is precisely this use case--the continuing evolution of a concrete and/or
% abstract semantics--where the analysis designer greatly benefit from using our
% framework.

% \begin{figure}
% \begin{lstlisting}
% type AEnv = Map String (Set AVal
% data AVal = IntAV | BoolAV | CloAV [String] Call
% 
% aop :: Op -> AVal -> Set AVal
% aop (Add1 IntAV) = Set.singleton IntAV
% aop (Sub1 IntAV) = Set.singleton IntAV
% aop (IsNonNeg IntAV) = Set.singleton BoolAV
% aop _ = Set.empty
% 
% aatomic :: Atom -> AEnv -> Set AVal
% aatomic (LitA (IntL _)) _ = Set.singleton IntAV
% aatomic (LitA (BoolL )) _ = Set.singleton BoolAV
% aatomic (VarA x) e = case Map.lookup x e of
%   Just avs -> avs
%   Nothing -> Set.empty
% aatomic (PrimA o a) _ = case aatomic a of
%   Just v -> aop o v
%   Nothing -> Set.empty
% aatomic (LamA x kx c) e = Set.singleton (CloAV [x, kx] c)
% aatomic (KonA x c) e = Set.singleton (CloAV [x] c e)
% 
% acall :: Call -> Env -> (Set Call, AEnv)
% acall (IfC a tc fc) e = 
%   if BoolAV `elem` aatomic a
%     then (Set.fromList [ tc, fb ], e)
%     else (Set.empty, e)
% acall (AppC fa xa ka) e =
%   merge (Set.empty, e) (concatMap (\ fv -> case fv of
%     CloAV [x, kx] c -> 
%       let e' = Map.unionWith Set.union (Map.singleton x (aatomic xv e))
%           e'' = Map.unionWith Set.union (Map.singleton kx (aatomic kx e'))
%       in [(c, e'')]
%     _ -> []) (Set.toList (aatomic fa)))
%   where
%     merge (cs, e) [] = (cs, e)
%     merge (cs, e) (c, e') = (Set.insert c cs, join e e')
%     
% acall (RetC ka xa) e =
%   Set.unions (Set.map (\ fv -> case fv of
%     CloAV [x] c -> 
%       let e' = Map.unionWith Set.union (Map.singleton x (aatomic xv e))
%       in Set.singleton (c, e')
%     _ -> Set.empty)) (aatomic fa)
% \end{lstlisting}
% \caption{``A Posteriori'' Abstract Semantics for CPS}
% \label{cps:abstract}
% \end{figure}


