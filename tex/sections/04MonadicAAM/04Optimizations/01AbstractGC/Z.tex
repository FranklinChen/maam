With the semantics in monadic style, abstract garbage collection can be
implemented once--generic to the underlying monad--integrating seamlessly into
both heap-cloning and heap-widening semantics.

%--%

To implement garbage collection we write a monad-generic function that crawls
the expression, environment and store, and replaces the store with a copy
containing only reachable addresses.
%
This implementation captures the generic garbage collection algorithm, and
scales to both heap-cloning and heap-widening stacks.
%
In fact, it's the exact algorithm used to garbage collect concrete semantics,
and behaves as concrete garbage collection when instantiated to concrete
semantics parameters.

%--$

Detecting free variables is a simple recursive function on the \h|Call|
expression in the state space.
%
\haskell{sections/04MonadicAAM/04Optimizations/01AbstractGC/00Free.hs}

%--%

Collecting the set of all reachable addresses demands a fixpoint computation,
iterating from the free variables as roots.
%
\haskell{sections/04MonadicAAM/04Optimizations/01AbstractGC/01Reachable.hs}

%--%

Abstract garbage collection replaces the existing heap with the abstract
garbage collected heap.
%
\haskell{sections/04MonadicAAM/04Optimizations/01AbstractGC/02GC.hs}
