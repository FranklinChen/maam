To give \h|Call| concrete semantics we first design a state space.
%
The state space introduces an environment to track variable bindings, and a
value type containing closures (which mutually close over environments).
%
\haskell{sections/03AAMByExample/01Concrete/00SS.hs}
%
The state space is partial (uses the \h|Maybe| type) and \h|Nothing| is the meaning
of failed computations.
%
Computations fail when an expression is ill-typed--for example if a literal
flows to function application position--or if a function application is of the
wrong arity.

%--%

The semantics for \h|Op| are given denotationally, and are straightforward.
%
\haskell{sections/03AAMByExample/01Concrete/01Op.hs}

%--%

The semantics for \h|Atom| are given denotationaly.
%
\haskell{sections/03AAMByExample/01Concrete/02Atom.hs}
%
Literals evaluate to themselves.
%
Variables evaluate to a value retrieved from the environment.
%
Lambdas evaluate immediately to closures which capture their environment.

%--%

The semantics for \h|Call| are given \textit{operationally} as a small step
function.
%
\haskell{sections/03AAMByExample/01Concrete/03Call.hs}
%
Conditional statements branch on boolean values.
%
Applications step to a function's body and closure environment with argument
values bound to formal parameters.
%
Termination is signaled with the \h|Halt| command.
%
Helper function \h|bindMany xs xas e| evaluates the function arguments \p|xas| and
binds them to the formal parameters \p|xs| in the environment \p|e|.
%
\haskell{sections/03AAMByExample/01Concrete/04BindMany.hs}
%
\h|bindMany xs xas e| fails if evaluating any argument in \p|xas| fails or if
there is an arity mismatch.

%--%

The full semantics of \h|Call| are given by the transitive closure of \h|step|.
%
\haskell{sections/03AAMByExample/01Concrete/05Step.hs}
