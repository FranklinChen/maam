To add state to our monad we apply a state monad transformer on top of the
current nondeterminism monad.
%
Monad transformers are compositional building blocks for building larger
monads.
%
At the interface level, monad transformers consume monads and produce monads;
they are mappings from monads to monads--monad morphisms.
%
Certain monad transformers are known to compose well with others.
%
In our case, nondeterminism and state are known to compose well (in this order
only), meaning the resulting monad will implement both state and nondeterminism
actions.

%--%

As we did with the nondeterminism monad and its interface \h|MonadPlus|, we
introduce an interface for state called \h|MonadState|.
%
\h|MonadState s m| means the monad \p|m| has access to a single cell of state
\p|s|.
%
\haskell{sections/04MonadicAAM/00MonadicStyle/01State/MonadState.hs}

%--%

The state monad transformer contains a function from state values to a monadic
actions of state-result pairs.
%
\haskell{sections/04MonadicAAM/00MonadicStyle/01State/01State.hs}
%
and implements the \h|MonadState| interface
%
\haskell{sections/04MonadicAAM/00MonadicStyle/01State/02StateMonadState.hs}

%--%

The benefit of monad transformers, and the reason why we bother to declare
interfaces for individual monads, is that monad transformer stacks have the
ability to expose the interfaces of inner monads to the fully composed outer
monad.
%
For example, \h|Nondet| has been shown to implement \h|MonadPlus|, but it is
also the case that the transformer stack \h|StateT s NonDet| \textit{also}
implements \h|MonadPlus|.
%
This lifting, or inheritance, of monadic interface can be expressed as a type
class instance rule.
%
\haskell{sections/04MonadicAAM/00MonadicStyle/01State/03StateMonadPlus.hs}

%--%

Our state space has \textit{three} state components, but the \h|MonadState|
interface only encodes one cell of state.
%
One solution is to combine each component of the state into a tuple, but this
would limit our ability to abstract each component in different ways.
%
A better solution is to allow multiple states in one stack, with a way of
selecting which state is desired.
%
The method for selecting the desired state can take many forms:
1) separate type classes could be defined for each state, 2) type singleton
tags (like the singletons used to fake modules in
section~\ref{section:AAMByExample}) could be defined to select the state, or 3)
we could require that each state is of unique type and allow type inference to
infer the desired state.
%
We take the approach of 2)--introducing type tags to select the desired state.
%
This approach adds a little more mechanism and type-trickery, but is much less
work than the 1) and avoids other newtype ``hacks'' required to use 3).

%--%

We redefine the \h|MonadState| class to use the \p|s| class parameter as a tag,
rather than the type of the state cell its self.
%
The actual type of the state is an associated type of the tag.
%
\haskell{sections/04MonadicAAM/00MonadicStyle/01State/04MonadStateTagged.hs}
%
The state type changes accordingly to store \h|Cell s|.
%
\haskell{sections/04MonadicAAM/00MonadicStyle/01State/05StateTagged.hs}
%
(The \h|Monad|, \h|MonadState| and \h|MonadPlus| instance are analagous.)

%--%

Now that we have a tagged state monad, we allow lifting of interior state
interfaces to the outside of a transformer stack.
%
\haskell{sections/04MonadicAAM/00MonadicStyle/01State/06MonadStateInterior.hs}

%--%

We now rewrite the semantics to use a four-layer monad transformer stack, which
we abbreviate to \h|M|.
%
The tags for the environment, store and time components of the monad are called
\h|E|, \h|S| and \h|T| and defined as type family instances.
%
\haskell{sections/04MonadicAAM/00MonadicStyle/01State/07M.hs}
%
The \h|atom| and \h|call| functions are then simplified to monadic actions.
%
\haskell{sections/04MonadicAAM/00MonadicStyle/01State/08Atom.hs}
%
\haskell{sections/04MonadicAAM/00MonadicStyle/01State/09Call.hs}

%--%

We are not finished until we also transform \h|step| to work with our new
monadic type.
%
The purpose of \h|step| is to translate the nondetermistic function \h|call|
into a state space transition.
%
We take this approach to heart and maintain a separation between the monadic
type and the type of its state space transitions.
%
Because the state space is separate from the monad, we reuse the same state
space from before.
%
\haskell{sections/04MonadicAAM/00MonadicStyle/01State/10StateSpace.hs}
%
The \h|step| function now converts between the monadic action 
\h|Call -> M Call| and state space transitions \h|SateSpace -> StateSpace|.
%
\haskell{sections/04MonadicAAM/00MonadicStyle/01State/11Step.hs}
