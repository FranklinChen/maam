The previous section demonstrated how to write the abstract semnatics from
section~\ref{section:AAMByExample:AbstractSemantics} in monadic style.
%
In this section we generalize the monad, allowing us to instantiate to
different monad stacks so long as they meet an interface.

%--%

The generalized interface will have parameters \p|d|, \p|aam| and \p|m| such
that:
%
\begin{itemize}
\item \h|Delta d| implements a value space primitive operations
\item \h|AAM aam| implements abstract address and time operations
\item \p|m| is a \h|Monad|, \h|MonadPlus| and \h|MonadState| for environment,
      store and time types.
\end{itemize}
%
This interface is encoded in full by the type class constraint 
\h|Semantics d aam m|.
%
\haskell{sections/04MonadicAAM/01Generalizing/00Semantics.hs}

%--%

We next manipulate the semantics into a form generic to parameters \p|d|,
\p|aam| and \p|m| such that \h|Semantics d aam m| holds.
%
\haskell{sections/04MonadicAAM/01Generalizing/01Atom.hs}
%
\haskell{sections/04MonadicAAM/01Generalizing/02Call.hs}
%
The code is omitted because it is identical to the code for the concrete \h|M|
in section~\ref{section:MonadicAAM:MonadicStyle}

%--%

Now we're in some trouble.
%
How do we implement \h|step| when we don't know what monad we're using?
%
The answer is to require the monad to come with an implementation of \h|step|.
%
Fortunately \h|step| can be implemented at the \textit{monad transformer}
level, so the implementation can be derived for an arbitrary monad transformer
stack.

%--%

We capture the ability for a monad to convert monadic actions to state space
transitions in another type class: \h|MonadStep|.
%
\haskell{sections/04MonadicAAM/01Generalizing/03MonadStep.hs}
%
\h|MonadStep| has associated type \h|SS m| for describing the type of step
functions \h|step :: SS m a -> SS m b|.
%
In our framework, we will always be instantiating step to 
\h|step :: SS m Call -> SS m Call|.
%
However this polymorphism is crucial to building state space transitions for
monad transformers, as they may instantiate the polymorphic type \p|a| to
something else.

%--%

We further require that the associated state space \h|SS m| be \h|Pointed| so
that we can inject pure values into the state space.
%
\haskell{sections/04MonadicAAM/01Generalizing/04Pointed.hs}

%--%

The only monad thus far, nondeterminism, is a \h|MonadStep| with its state
space \h|SS| being list and \h|step| being simply monadic bind.
%
\haskell{sections/04MonadicAAM/01Generalizing/05NondetMonadStep.hs}
%
The only monad transformer thus far, state, is a \h|MonadStep| transformer as
well, taking \h|MonadStep| things to \h|MonadStep| things.
%
\haskell{sections/04MonadicAAM/01Generalizing/06StateTMonadStep.hs}
%
Note that in order for the state space associated with \h|StateT| to be
\h|Pointed| we require \h|Cell s| to be a \h|JoinLattice|, meaning it has a
bottom element.

%--%

Executing the semantics can be achieved by injecting the initial \h|Call| value
into the state space and iterating the collecting function as before.
%
\haskell{sections/04MonadicAAM/01Generalizing/07Exec.hs}
