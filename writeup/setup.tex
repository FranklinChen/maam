\usepackage{amsmath}
\usepackage{unicode-math}
\setmainfont{XITS}
\setmathfont{XITS Math}
\usepackage{tikz}
\usetikzlibrary{matrix}
\usepackage{newunicodechar}
\usepackage{galois}
% \usepackage{multicol}

% \newunicodechar{λ}{\lambda}
% \newunicodechar{→}{\rightarrow}
% \newunicodechar{𝒯}{\mathcal{T}}
% \newunicodechar{∀}{\forall}
% \newunicodechar{∘}{\circ}
% \newunicodechar{𝒫}{𝒫\ }


\newcommand{\iif}{\underline{\text{if}}}
\newcommand{\case}{\underline{\text{case}}}
\newcommand{\halt}{\underline{\text{halt}}}
\newcommand{\lam}{\underline{\text{λ}}}
\newcommand{\add}{\text{add1}}
\newcommand{\sub}{\text{sub1}}
\newcommand{\gez}{\text{gez}}
\newcommand{\INT}{\text{INT}}
\newcommand{\TRUE}{\text{TRUE}}
\newcommand{\FALSE}{\text{FALSE}}
\newcommand{\ddo}{\operatorname{do}}
\newcommand{\llet}{\operatorname{let}}
\newcommand{\return}{\operatorname{return}}
\newcommand{\getstore}{\operatorname{get-σ}}
\newcommand{\modifystore}{\operatorname{modify-σ}}
\newcommand{\coercebool}{\operatorname{↓𝔹}}
\newcommand{\coercekon}{\operatorname{↓λ₁}}
\newcommand{\coercefun}{\operatorname{↓λ₂}}
\newcommand{\liftpowerset}{\operatorname{↑𝒫}}
\newcommand{\touchedcall}{\operatorname{𝓉𝒞}}
\newcommand{\touchedvar}{\operatorname{𝓉Var}}

\newcommand{\C}{L}
\newcommand{\A}{\widehat{L}}

\newcommand{\Csteps}{↦}
\newcommand{\Asteps}{\ \widehat{↦}\ }

\newcommand{\Ce}{e}
\newcommand{\Ae}{\widehat{e}}

\newcommand{\AStore}{\widehat{Store}}
\newcommand{\AVal}{\widehat{Val}}

\newcommand{\PM}{𝒫\ }
\newcommand{\PT}{𝒫_T\ }

\newcommand{\SM}{State}
\newcommand{\ST}{State_T}

\newenvironment{donotbreak}
{\\[0.5\baselineskip]\begin{minipage}{\linewidth}}
{\end{minipage}\\[0.5\baselineskip]}

% \usepackage{amssymb}
% \usepackage{listings}
% \usepackage{stmaryrd}
% \usepackage{wasysym}
% \usepackage{cite}
% \usepackage{pgfplots}
% \usepackage{minted}
% \usepackage{float}
% \usepackage{caption}
% \usepackage{fontspec}
% \usepackage{amsthm}
% \setmonofont{Inconsolata}
% % \usepackage{hyperref}
% 
% \lstset{basicstyle=\footnotesize\ttfamily,language=haskell}
% \lstMakeShortInline|
% 
% \newmintedfile[haskell]{haskell}{}
% \newmintinline[h]{haskell}{}
% \newmintinline[p]{text}{}
% 
% \newenvironment{itemizenobreak}
% {\\[0.5\baselineskip]\begin{minipage}{\linewidth}\begin{itemize}}
% {\end{itemize}\end{minipage}\\[0.5\baselineskip]}
% 
% 
% \newcommand{\DSS}{\text{\lstinline|SS|}}
% \newcommand{\DSSAbs}{\widehat{\text{\lstinline|SS|}}}
% \newcommand{\Dstep}{\text{\lstinline|step|}}
% \newcommand{\DstepAbs}{\widehat{\text{\lstinline|step|}}}
% 
% %%% pulled from internet (stack overflow user Christian Feuersänger) %%%
% % argument #1: any options
% \newenvironment{customlegend}[1][]{%
%     \begingroup
%     % inits/clears the lists (which might be populated from previous
%     % axes):
%     \csname pgfplots@init@cleared@structures\endcsname
%     \pgfplotsset{#1}%
% }{%
%     % draws the legend:
%     \csname pgfplots@createlegend\endcsname
%     \endgroup
% }%
% 
% % makes \addlegendimage available (typically only available within an
% % axis environment):
% \def\addlegendimage{\csname pgfplots@addlegendimage\endcsname}
% 
