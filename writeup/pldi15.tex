%lambdlambdaa any author declaration will be ignored  when using 'plid' option (for double blind review)
\documentclass[pldi,nocopyrightspace]{sigplanconf}

% From PLDI Template {-{

% \usepackage{SIunits}            % typset units correctly
\usepackage{courier}            % standard fixed width font
\usepackage[scaled]{helvet} % see www.ctan.org/get/macros/latex/required/psnfss/psnfss2e.pdf
\usepackage{url}                  % format URLs
\usepackage{listings}          % format code
\usepackage{enumitem}      % adjust spacing in enums
\usepackage{amsmath}
\usepackage[colorlinks=true,allcolors=blue,breaklinks,draft=false]{hyperref}   % hyperlinks, including DOIs and URLs in bibliography
% known bug: http://tex.stackexchange.com/questions/1522/pdfendlink-ended-up-in-different-nesting-level-than-pdfstartlink
\newcommand{\doi}[1]{doi:~\href{http://dx.doi.org/#1}{\Hurl{#1}}}   % print a hyperlinked DOI

% }-}

% Setup {-{

\usepackage{amsmath}
\usepackage{amsthm}
\usepackage{amsfonts}
\usepackage{amssymb}
\usepackage{mathtools}
\usepackage{bm}
\usepackage{stmaryrd}
\usepackage{galois}
\usepackage{float}
\usepackage{latexsym}
\floatstyle{boxed}
\restylefloat{figure}

\newtheorem{proposition}{Proposition}
\newtheorem{corollary}{Corollary}
\newtheorem{definition}{Definition}

% This makes newlines not introduce paragraphs.
% Paragraphs must be explicitly marked with \par
% WHAT WIZARDRY IS THIS
\endlinechar=9\relax

% }-}

% Macros {-{

\newcommand\litM{M}
\newcommand\litTime{Time}
\newcommand\litStore{Store}
\newcommand\litVal{Val}
\newcommand\litKAddr{KAddr}
\newcommand\litKStore{KStore}
\newcommand\litEnv{Env}
\newcommand\litClo{Clo}

\newcommand{\itop}[1]{\operatorname{\mathit{#1}}}
\newcommand{\ttbfop}[1]{\operatorname{\mathtt{\boldsymbol{\mathbf{#1}}}}}
\newcommand{\ttbfbin}[1]{\mathbin{\mathtt{\mathbf{\boldsymbol{#1}}}}}
\newcommand{\ttop}[1]{\operatorname{\mathtt{#1}}}
\newcommand{\Concrete}[1]{\mathtt{\mathbf{#1}}}
\newcommand{\Abstract}[1]{\widehat{\mathtt{\mathbf{#1}}}}

% }-}

\begin{document}

% These are supposed to make align* not have breaks around it...
\setlength{\abovedisplayskip}{0em}
\setlength{\belowdisplayskip}{0em}
\setlength{\abovedisplayshortskip}{0em}
\setlength{\belowdisplayshortskip}{0em}

%\title{The Marriage of Monad Transformers and Abstract Interpreters: Modular Metatheory for Program Analysis}
\title{Galois Transformers and Modular Abstract Interpreters:\\ Reusable Metatheory for Program Analysis}
\authorinfo{David Darais}{Harvard University, USA}{darais@seas.harvard.edu}
\authorinfo{Matthew Might}{University of Utah, USA}{might@cs.utah.edu}
\authorinfo{David Van Horn}{University of Maryland, USA}{dvanhorn@cs.umd.edu}
\maketitle

% Abstract {-{
\begin{abstract}

The design and implementation of static analyzers have becoming
increasingly systematic.  In fact, design and implementation have
remained seemingly on the verge of full mechanization for several
years.  A stumbling block in full mechanization has been the ad hoc
nature of soundness proofs accompanying each analyzer.  While design
and implementation is largely systematic, soundness proofs can change
significantly with (apparently) minor changes to the semantics and
analyzers themselves.

\par We solve the problem of systematically constructing static
analyzers by introducing the notion of a \emph{Galois transformer}, a
monad transformer that satisfies the properties of a Galois
connection.  In concert with a monadic interpreter, we are able to
define a library of monad transformers implementing classic context,
path, and heap (in-)sensitive building blocks which can be composed
together to realize a program analysis monad, \emph{independent of the
  language being analyzed}.

\par Significantly, a Galois transformer can be proved sound once and
for all, making it a reusable analysis component.  As new analysis
features and abstractions are developed and mixed in, soundness proofs
need not be reconstructed, as the composition of a monad transfomer
stack is sound by virtue of its consituents.  We believe Galois
transformers provide a viable foundation for resuable and composable 
metatheory for program analysis.
\end{abstract}
% }-}

\input{tmp/autogen/pldi15.markdown.tex}

\bibliographystyle{abbrvnat}
\bibliography{dvanhorn,davdar}

\end{document}
