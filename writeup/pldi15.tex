% any author declaration will be ignored  when using 'plid' option (for double blind review)
\documentclass[pldi]{sigplanconf}

% From PLDI Template {-{

% \usepackage{SIunits}            % typset units correctly
\usepackage{courier}            % standard fixed width font
\usepackage[scaled]{helvet} % see www.ctan.org/get/macros/latex/required/psnfss/psnfss2e.pdf
\usepackage{url}                  % format URLs
\usepackage{listings}          % format code
\usepackage{enumitem}      % adjust spacing in enums
\usepackage[colorlinks=true,allcolors=blue,breaklinks,draft=false]{hyperref}   % hyperlinks, including DOIs and URLs in bibliography
% known bug: http://tex.stackexchange.com/questions/1522/pdfendlink-ended-up-in-different-nesting-level-than-pdfstartlink
\newcommand{\doi}[1]{doi:~\href{http://dx.doi.org/#1}{\Hurl{#1}}}   % print a hyperlinked DOI

% }-}

% Setup {-{

\usepackage{fontspec}
\setmainfont{CMU Serif}   % unicode variant of CM

\usepackage{amsmath}      % \begin{align*}
\usepackage{amssymb}      % \leadsto
\usepackage{bm}           % bold maths
\usepackage{unicode-math} % unicode maths
\usepackage{galois}       % \galois

\usepackage{float}        % \begin{figure}
\floatstyle{boxed}
\restylefloat{figure}

% }-}

% Macros {-{

\newcommand{\lamif}{\textbf{λ-IF}\@}

\newcommand{\ifz}[3]{\pmb{\operatorname{if0}} (#1)\{#2\}\{#3\}}
\newcommand{\lam}{\pmb{λ}}
\newcommand{\indor}{\;|\;}
\newcommand{\plus}{\pmb{+}}
\newcommand{\minus}{\pmb{-}}
\newcommand{\op}{\mathbin{op}}
\newcommand{\ap}{\;\;}
\newcommand{\steps}{\leadsto}
\newcommand{\ddo}{\underline{\operatorname{do}}}
\newcommand{\case}[1]{\text{\underline{case} $#1$ \underline{of}}}
\newcommand{\pushkon}{\operatorname{push-Kon}}
\newcommand{\popkon}{\operatorname{pop-Kon}}
\newcommand{\getenv}{\operatorname{get-Env}}
\newcommand{\putenv}{\operatorname{put-Env}}
\newcommand{\ret}{\underline{\operatorname{return}}}
\newcommand{\toclo}{\operatorname{to-clo}}
\newcommand{\mif}[3]{\text{\underline{if} $#1$ \underline{then} $#2$ \underline{else} $#3$}}
\newcommand{\stepf}{\operatorname{step}}
\newcommand{\atomf}{\operatorname{atom}}
\newcommand{\branchOnSet}{\operatorname{branch-on-set}}
\newcommand{\intI}{\operatorname{int-I}}
\newcommand{\ifzE}{\operatorname{if0-E}}
\newcommand{\cloI}{\operatorname{clo-I}}
\newcommand{\cloE}{\operatorname{clo-E}}
\newcommand{\st}{\;|\;}

% }-}

\begin{document}

\title{Modular Static Analysis: Orthogonal Design, Correct by Construction}
\maketitle

% Abstract {-{
% \begin{abstract}
% \end{abstract}
% }-}

% Introduction {-{
% \section{Introduction}

% }-}

% Modular Analysis by Example {{{
\section{Modular Analysis by Example}

To demonstrate our framework we will grow a concrete semantics into an executable analysis in small steps.
Each step will expose some property of the analysis to be tuned independent of other properties.
We use an applied lambda calculus as an example language to demonstrate our approach.

% }}}

% Concrete Semantics {-{
\subsection{Concrete Semantics}

Our example language is \lamif, an applied lambda caluclus with integers and conditional statements.
The syntax for \lamif\ is given in figure~\ref{lamif:syntax}.

\begin{figure}[t]
\caption{\lamif\ syntax.}
\label{lamif:syntax}
\begin{alignat*}{2}
   x &∈ Var &&                                                                         \\
   i &∈ ℤ   &&                                                                         \\
  op &∈ Op  &&::= \plus \indor \minus                                                  \\
   a &∈ Atom &&::= x \indor i \indor \lam(x) → e                                       \\
   e &∈ Exp &&::= a \indor (e₁ \op e₂) \indor (e₁ \ap e₂) \indor \ifz{e₁}{e₂}{e₃}
\end{alignat*}

\end{figure}

We will use a standard small-step semantics for \lamif, which is shown in figure~\ref{lamif:semantics}.

\begin{figure}[t]
\caption{\lamif\ semantics.}
\label{lamif:semantics}
{\scriptsize
\begin{alignat*}{2}
          clo &∈ Clo     &&:= ⟨\lam(x) → e, ρ⟩                                           \\
            v &∈ Val     &&::= i \indor clo                                              \\
            ρ &∈ Env     &&:= Var ⇀ Val                                                  \\
           fr &∈ Frame   &&::= [ □ \op e ] \indor [ v \op □ ] 
                        \indor [ □ \ap e ] \indor [ v \ap □ ] \indor [ \ifz{□}{e₁}{e₂} ] \\
\overline{fr} &∈ Kon     &&:= Frame^*                                                    \\
            ς &∈ Σ       &&:= Exp × Env × Kon                                        \\
       \steps &∈ 𝒫 (Σ × Σ)
\end{alignat*}

\begin{alignat*}{1}
  ⟨ a , ρ , [ □ \op e ] ∷ \overline{fr} ⟩ &\steps ⟨ e                         , ρ , [ 𝒜_ρ⟦a⟧ \op □ ] ∷ \overline{fr} ⟩         \\
  ⟨ a , ρ , [ v \op □ ] ∷ \overline{fr} ⟩ &\steps ⟨ \delta(\op , v , 𝒜_ρ⟦a⟧ ) , ρ , \overline{fr} ⟩                            \\
%
  ⟨ a , ρ , [ □                   \ap e ] ∷ \overline{fr} ⟩ &\steps ⟨ e , ρ               , [ 𝒜_ρ⟦a⟧ \ap □ ] ∷ \overline{fr} ⟩ \\
  ⟨ a , ρ , [ ⟨ \lam(x) → e, ρ' ⟩ \ap □ ] ∷ \overline{fr} ⟩ &\steps ⟨ e , ρ'[x ↦ 𝒜_ρ⟦a⟧ ] , \overline{fr} ⟩                    \\
%
  ⟨ a , ρ , [ \ifz{□}{e₁}{e₂} ] ∷ \overline{fr} ⟩ &\steps 
  \begin{cases} 
      ⟨ e₁ , ρ , \overline{fr} ⟩ & \text{if $𝒜_ρ⟦a⟧ = 0$}            \\
      ⟨ e₂ , ρ , \overline{fr} ⟩ & \text{otherwise} 
  \end{cases} \\
%
  ⟨ e₁ \op e₂        , ρ , \overline{fr} ⟩ &\steps ⟨ e₁   , ρ , [ □ \op e₂ ]        ∷ \overline{fr} ⟩               \\
  ⟨ e₁ \ap e₂        , ρ , \overline{fr} ⟩ &\steps ⟨ e₁   , ρ , [ □ \ap e₂ ]        ∷ \overline{fr} ⟩               \\
  ⟨ \ifz{e₁}{e₂}{e₃} , ρ , \overline{fr} ⟩ &\steps ⟨ e₁   , ρ , [ \ifz{□}{e₂}{e₃} ] ∷ \overline{fr} ⟩
\end{alignat*}

}
\end{figure}

The analyses we will consider will be stated as the least fixed point of some abstract collecting semantics.
A collecting semantics for \lamif\ is defined as:
\begin{equation*}
  μ(ς). ς₀ ⊔ ς ⊔ \{ ς' \st ς \steps ς' \}
\end{equation*}
where $ς₀ := ⟨ e₀ , ⊥ , ∙ ⟩$ is the injection of the initial program $e₀$ into the state space $Σ$.

% }-}

% From Semantics to Interpreter {-{
\section{From Semantics to Interpreter}

On our way to an executable analysis for \lamif, we first evolve the small-step relation into an executable small-step interpreter.
We write this interpreter in monadic style for two reasons.
First, it is much easier to add new states and rules to a monadic interpreter.
Second, we will exploit the monadic abstraction to expose flow and path sensitivities in a later interpreter.

The monad for the interpreter will have two types effects: state and partiality.
Two operations carry state effects: get and put; and one operation carries a partiality effect: fail.

We construct a monad $ℳ $ which combines state effects for $Env$ and $Kon$ with partiality effects.
The type $ℳ $ is just a simple type which supports the monad, state, and partiality operations.
The interpreter for \lamif\ which uses $ℳ $ and monadic effects is given in figure~\ref{lamif:minterpreter}.

\begin{figure}[t]
\caption{\lamif\ monadic interpreter.}
\label{lamif:minterpreter}
{\scriptsize
\begin{alignat*}{1}
& \hspace{0em}   ℳ (a) := Env × Kon → {(a × Env × Kon)}_⊥                      \\
& \hspace{0em}   \atomf : Atom → ℳ (Exp)                                       \\
& \hspace{0em}   \atomf(a) := \ddo                                             \\
& \hspace{2em}     v ← 𝒜 ⟦a⟧                                                   \\
& \hspace{2em}     fr ← \popkon                                                \\
& \hspace{2em}     \case{fr}                                                   \\
& \hspace{4em}       [ □  \op e ] → \ddo                                       \\
& \hspace{6em}         \pushkon([ v \op □ ])                                   \\
& \hspace{6em}         \ret(e)                                                 \\
& \hspace{4em}       [ v' \op □ ] → \ret(\delta(\op, v', v))                   \\
& \hspace{4em}       [ □  \ap e ] → \ddo                                       \\
& \hspace{6em}         \pushkon([ v \ap □ ])                                   \\
& \hspace{6em}         \ret(e)                                                 \\
& \hspace{4em}       [ v' \ap □ ] → \ddo                                       \\
& \hspace{6em}         ⟨ \lam(x) → e, ρ' ⟩ ← \toclo(v')                        \\
& \hspace{6em}         \putenv(ρ'[x ↦ v])                                      \\
& \hspace{6em}         \ret(e)                                                 \\
& \hspace{4em}       [ \ifz{□}{e₁}{e₂} ] → \mif{v ≟ 0}{\ret(e₁)}{\ret(e₂)}     \\
& \hspace{0em}   \stepf : Exp → ℳ (Exp)                                        \\
& \hspace{0em}   \stepf(a) := \atomf(a)                                        \\
& \hspace{0em}   \stepf(e₁ \op e₂) := \ddo                                     \\
& \hspace{2em}     \pushkon([ □ \op e₂ ])                                      \\
& \hspace{2em}     \ret(e₁)                                                    \\
& \hspace{0em}   \stepf(e₁ \ap e₂) := \ddo                                     \\
& \hspace{2em}     \pushkon([ □ \ap e₂ ])                                      \\
& \hspace{2em}     \ret(e₁)                                                    \\
& \hspace{0em}   \stepf(\ifz{e₁}{e₂}{e₃}) := \ddo                              \\
& \hspace{2em}     \pushkon([\ifz{□}{e₂}{e₃}])                                 \\
& \hspace{2em}     \ret(e₁)
\end{alignat*}

}
\end{figure}

As before, we must construct a collecting semantics in order to define an executable analysis.
To do this, we must relate our step function back to some state machine step \emph{function}.
One we c % HERE

% }-}

% Exposing the Abstract Domain {-{
\section{Exposing the Abstract Domain}

We will now expose the choice of abstract domain for the interpreter.
However, giving a finite abstract domain for integers alone will not yet give a computable analysis.
There will still be an infinite number of possible abstract state spaces due to the recursion between closures and environments.
We will address this issue in a future modification to the interpreter.

To expose the choice of abstract domain we introduce a type $AVal$ behind an abstract interface.
\begin{alignat*}{1}
    AVal &∈ Set                \\
   \intI &: ℤ → AVal           \\
   \ifzE &: AVal → 𝒫 (𝔹)       \\
   \cloI &: Clo → AVal         \\
   \cloE &: AVal → 𝒫 (Clo)     \\
  \delta &: Op → AVal → AVal → AVal
\end{alignat*}

To establish the correctness of our interpreter $\stepf$, we must prove that it is monotonic in $AVal$, $intI$, $if0E$, $cloI$ and $cloE$.
This can be done independent of a specific $AVal$.
Then we need only establish $Val \galois{α}{γ} AVal$ for some $AVal$ in order to justify the correctness of $\stepf$ instantiated with $AVal$.
  
Exposing the abstract domain forces us to reconsider our monadic effects.
For example, a branch on a possibly unknown integer value must return multiple possible machine states.
Therefore we must trade our partiality monad for a nonteterminism monad.
The monadic abstraction supports nondeterminism nicely in this setting.
The helper function $\branchOnSet$ will return multiple times for each value in the set.
This allows the interpreter to be written in a style that need only consider one state at a time.
The monadic interpreter whith the abstract domain exposd is given in figure~\ref{lamif:minterpreter:adom}.
(The implementation of $\stepf$ is unchanged.)

\begin{figure}[t]
\caption{\lamif\ monadic interpreter with abstract domain exposed.}
\label{lamif:minterpreter:adom}
{\scriptsize
\begin{alignat*}{1}
& \hspace{0em}   AEnv := Var → AVal                                            \\
& \hspace{0em}   ℳ (a) := AEnv × Kon → 𝒫 (a × AEnv × Kon)                      \\
& \hspace{0em}   \atomf : Atom → ℳ (Exp)                                       \\
& \hspace{0em}   \atomf(a) := \ddo                                             \\
& \hspace{2em}     v ← 𝒜 ⟦a⟧                                                   \\
& \hspace{2em}     fr ← \popkon                                                \\
& \hspace{2em}     \case{fr}                                                   \\
& \hspace{4em}       [ □  \op e ] → \ddo                                       \\
& \hspace{6em}         \pushkon([ v \op □ ])                                   \\
& \hspace{6em}         \ret(e)                                                 \\
& \hspace{4em}       [ v' \op □ ] → \ret(\delta(\op, v', v))                   \\
& \hspace{4em}       [ □  \ap e ] → \ddo                                       \\
& \hspace{6em}         \pushkon([ v \ap □ ])                                   \\
& \hspace{6em}         \ret(e)                                                 \\
& \hspace{4em}       [ v' \ap □ ] → \ddo                                       \\
& \hspace{6em}         ⟨ \lam(x) → e, ρ' ⟩ ← \branchOnSet(\cloE(v'))           \\
& \hspace{6em}         \putenv(ρ'[x ↦ v])                                      \\
& \hspace{6em}         \ret(e)                                                 \\
& \hspace{4em}       [ \ifz{□}{e₁}{e₂} ] → \ddo                                \\
& \hspace{6em}         b ← \branchOnSet(\ifzE(v))                              \\
& \hspace{6em}         \mif{v ≟ 0}{\ret(e₁)}{\ret(e₂)}                         \\
\end{alignat*}

}
\end{figure}

% }-}

\end{document}
