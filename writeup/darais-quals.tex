\documentclass{article}

\usepackage{amsmath}
\usepackage{unicode-math}
\setmainfont{XITS}
\setmathfont{XITS Math}
\usepackage{tikz}
\usetikzlibrary{matrix}
\usepackage{newunicodechar}
\usepackage{galois}
\usepackage{amsthm}

\theoremstyle{definition}
\newtheorem*{definition}{Definition}
\newtheorem*{example}{Example}

\theoremstyle{plain}
\newtheorem*{lemma}{Lemma}
\newtheorem{theorem}{Theorem}
\newtheorem{corrolary}{Corrolary}

\newcommand{\iif}{\underline{\text{if}}}
\newcommand{\case}{\underline{\text{case}}}
\newcommand{\halt}{\underline{\text{halt}}}
\newcommand{\lam}{\underline{\text{λ}}}
\newcommand{\add}{\text{add1}}
\newcommand{\sub}{\text{sub1}}
\newcommand{\gez}{\text{gez}}
\newcommand{\INT}{\text{INT}}
\newcommand{\TRUE}{\text{TRUE}}
\newcommand{\FALSE}{\text{FALSE}}
\newcommand{\ddo}{\operatorname{do}}
\newcommand{\llet}{\operatorname{let}}
\newcommand{\return}{\operatorname{return}}
\newcommand{\getstore}{\operatorname{get-σ}}
\newcommand{\putstore}{\operatorname{put-σ}}
\newcommand{\getenv}{\operatorname{get-ρ}}
\newcommand{\putenv}{\operatorname{put-ρ}}
\newcommand{\gettime}{\operatorname{get-τ}}
\newcommand{\puttime}{\operatorname{put-τ}}
\newcommand{\coercebool}{\operatorname{↓𝔹}}
\newcommand{\coercekon}{\operatorname{↓λ₁}}
\newcommand{\coercefun}{\operatorname{↓λ₂}}
\newcommand{\liftpowerset}{\operatorname{↑𝒫}}
\newcommand{\touchedcall}{\operatorname{𝓉𝒞}}
\newcommand{\touchedvar}{\operatorname{𝓉Var}}

\newcommand{\C}{L}
\newcommand{\A}{\widehat{L}}

\newcommand{\Csteps}{↦}
\newcommand{\Asteps}{\ \widehat{↦}\ }

\newcommand{\Ce}{e}
\newcommand{\Ae}{\widehat{e}}

\newcommand{\AStore}{\widehat{Store}}
\newcommand{\AEnv}{\widehat{Env}}
\newcommand{\ATime}{\widehat{Time}}
\newcommand{\AVal}{\widehat{Val}}

\newcommand{\PM}{𝒫\ }
\newcommand{\PT}{𝒫_T\ }

\newcommand{\SM}{State}
\newcommand{\ST}{State_T}

\newenvironment{donotbreak}
{\\[0.5\baselineskip]\begin{minipage}{\linewidth}}
{\end{minipage}\\[0.5\baselineskip]}

% \usepackage{multicol}

% \newunicodechar{λ}{\lambda}
% \newunicodechar{→}{\rightarrow}
% \newunicodechar{𝒯}{\mathcal{T}}
% \newunicodechar{∀}{\forall}
% \newunicodechar{∘}{\circ}
% \newunicodechar{𝒫}{𝒫\ }


% \usepackage{amssymb}
% \usepackage{listings}
% \usepackage{stmaryrd}
% \usepackage{wasysym}
% \usepackage{cite}
% \usepackage{pgfplots}
% \usepackage{minted}
% \usepackage{float}
% \usepackage{caption}
% \usepackage{fontspec}
% \setmonofont{Inconsolata}
% % \usepackage{hyperref}
% 
% \lstset{basicstyle=\footnotesize\ttfamily,language=haskell}
% \lstMakeShortInline|
% 
% \newmintedfile[haskell]{haskell}{}
% \newmintinline[h]{haskell}{}
% \newmintinline[p]{text}{}
% 
% \newenvironment{itemizenobreak}
% {\\[0.5\baselineskip]\begin{minipage}{\linewidth}\begin{itemize}}
% {\end{itemize}\end{minipage}\\[0.5\baselineskip]}
% 
% 
% \newcommand{\DSS}{\text{\lstinline|SS|}}
% \newcommand{\DSSAbs}{\widehat{\text{\lstinline|SS|}}}
% \newcommand{\Dstep}{\text{\lstinline|step|}}
% \newcommand{\DstepAbs}{\widehat{\text{\lstinline|step|}}}
% 
% %%% pulled from internet (stack overflow user Christian Feuersänger) %%%
% % argument #1: any options
% \newenvironment{customlegend}[1][]{%
%     \begingroup
%     % inits/clears the lists (which might be populated from previous
%     % axes):
%     \csname pgfplots@init@cleared@structures\endcsname
%     \pgfplotsset{#1}%
% }{%
%     % draws the legend:
%     \csname pgfplots@createlegend\endcsname
%     \endgroup
% }%
% 
% % makes \addlegendimage available (typically only available within an
% % axis environment):
% \def\addlegendimage{\csname pgfplots@addlegendimage\endcsname}
% 


\title{Abstract Control in Static Analysis}
\author{David Darais}
\date{\today}

\begin{document}
\maketitle

\begin{abstract} % {{{
\end{abstract} % }}}

\tableofcontents

% Introduction {{{
\section{Introduction}
\label{section:Introduction}

Static analyses have two components: 
%
\begin{itemize}
\item A \emph{computational artifact} which computes a set of facts about the execution of a program.
\item A \emph{proof of correctness} about the computational artifact.
\end{itemize}


We are motivated by static analysis techniques which are:
%
\begin{itemize}
\item Compartmental: Various aspects of the analysis have been separated from each other and are understood in isolation.
\item Modular: Design choices in one aspect of the analysis do not restrict design choices of another.  This property is important for both computational and correctness components of an analysis.
\item Language Agnostic: A given analysis technique can be seamlessly transfered from one semantics to another.
\item Correct: The proof of correctness should exist, and there should be a framework for establishing it.
\item Derived: Either the computational artifact or the correctness proof should be obtained "for free" from the other.
\end{itemize}

Our framework compartmentalizes:
%
\begin{itemize}
\item Abstract domain a la Cousot.
\item Abstract time and address a la Van Horn and Might.
\item Intensional optimizations a la Van Horn and Might. 
\item Object-sensitivity a la Smaragdakis.
\item Analysis control properties.
\end{itemize}


Our framework is:
%
\begin{itemize}
\item Compartmental: Each of the above concerns are separated and independent.
\item Modular: Design choices in one axis is completely independent of others
\item Language Agnostic: All axis other than abstract domain are fully language agnostic.
\item Correct: A correctness framework is described and each axis is proven correct in isolation.
\item Derived: The correctness of an analysis using our framework comes for free.
\end{itemize}


Analysis control properties are the primary topic of this work.
%
We merely note that the other axis in the design space are implemented in our framework using existing techniques.


Contributions:
%
\begin{itemize}
\item Computational artifacts and correctness proofs for the axis of abstract control, which captures the choice of flow and path sensitivity.
\item Computational artifacts and correctness proofs for intensional optimizations (gargabe collection and mcfa) independent of abstract control.
\item Language independent proofs of refinement for flow-sensitivity and path-sensitivity choices using the monadic abstraction.
\end{itemize}
 
% }}}

% Background {{{
\section{Background}
\label{section:Background}
 
% Notation {{{
\subsection{Notation}
\label{section:Background:Notation}


We use traditional notation for function definitions and applications rather than lambda calculus notation.
%
For example, we will define [f(w, x) := g(y, z)] rather than [f w x := g y z].


We use lambda notation for anonymous functions.
%
For example, defining the identify function could be notated [id := λ(x) → x] rather than the named definition [id(x) := x].


% }}}

% Monads {{{
\subsection{Monads}
\label{section:Background:Monads}

A basic familiarity with the concept of monads will be useful for understanding the rest of the paper.
%
Some find it useful to review the concept of functor before learning about monads. 
%
We adopt this approach for our brief explanation of monads.


A type [𝒯 : Set → Set] is called a functor if one can:
%
\begin{itemize}
\item define [map : ∀ (A : Set), (A → B) → 𝒯(A) → 𝒯(B)]
\item prove [map id = id], where id is the identity function [λ x → x]
\item prove [map (g ∘ f) = map g ∘ map f]
\end{itemize}


Example: Lists are functors, where map is defined:
%
    map(f)(xs) := case xs 
      Nil → Nil 
      Cons(x,xs') → Cons(f(x),map(f)(xs'))


Example: [map(isEven)([1, 2, 3, 4]) = [False, True, False, True]]


Likewise a type [𝒯 : Set → Set] is called a monad if one can:
%
\begin{itemize}
\item define [extend : ∀ (A : Set), (A → 𝒯(B)) → 𝒯(A) → 𝒯(B))]
\item prove unit and associativity laws (not mentioned here).
\end{itemize}


Example: Lists are monads, where extend is defined:
%
    extend(f)(xs) := case xs
      Nil → Nil
      Cons(x,xs') → f(x) ++ extend(f)(xs')
%
[extend] for lists can be understood as a map followed by a concat or flattening operation.


Example: [extend(λ x → [x + 1, x - 1])([10, 100]) = [9, 11, 99, 101]


[extend] for lists can also be understood as concat followed by map:
%
      extend(f)([10, 100]) 
    = concat(map(f)([10, 100])) 
    = concat([[9, 11], [99, 101]]) = [9, 11, 99, 101]
%
For all monads, it is equivalent to define return, map and join, where [join :
∀ (A : Set), 𝒯(𝒯(A)) → 𝒯(A)].
%
As we have seen, join is just concat for the list monad.


% }}}

% Small Step Semantics {{{
\subsection{Small-Step Semantics}
\label{section:Background:SmallStepSemantics}

This work follows the small-step operational approach to language semantics pioneered by ? and popularized by ?.
%
In small-step operational semantics, the meaning of an expression ℯ in a language is given by the transitive closure of a step relation [ℯ ↦ ℯ'].
%
This yields a state-machine approach to semantics.
%
Exploring the meaning of the expression means exploring all possible transitions reachable from ℯ through ↦.

% }}}

% Abstract Interpretation {{{
\subsection{Abstract Interpretation}
\label{section:Background:AbstractInterpretation}

Abstract Iterpretation (AI) is a formal framework for program analysis pioneered by Cousot et Cousot.
%
In the setting of abstract interpretation, a program analysis is an alternate semantics over an abstract domain.

both a discipline for constructing program analyses
and a framework for reasoning about their correctness pioneered by Cousot and
Cousot. Van Horn and Might have extended this tradition in their work on
Abstracting Abstract Machines, showing how to systematically construct analysis
for small step operational semantics.
%
Relating an analysis to a semantics requires a bridge.  In general, such a
relationship will be stated as:
%
∀ (c c': C) (a a': A), c R a ∧ c ↦ c' ∧ a ↦ a' ⇒  c' R a'
%
In words, this is saying that for a concrete program c and a valid abstraction
of it a, if the program can take a step to c' and the abstraction can take a
step to a', then a' is a valid abstraction of c'.
%
Once this theorem is established, the transitive closure of it follows
trivially and justifies performing an analysis by:
%
1: abstracting a program c to it its abstraction a
2: running a using the transitive closure of a ↦* a'
3: inspecting the final a' to infer behavior of running c
%
The question now is how to set up R--how to relate one language and semantics
to another. For this, abstract interpretation turns to order theory.  We say c
and a are related if c is in the image of projecting a to 𝒫(c).  Using this
projection, commonly called γ, we restate the theorem:
%
∀ (c c': C) (a a': A), c ∈ γ(a) ∧ c ↦ c' ∧ a ↦ a' ⇒  c' ∈ γ(a')
%
However, the technique generalizes to arbitrary partial orders, and is done so
profitably in the literature.  Stating the theorem with partial orders gives:
%
∀ (c c': C) (a a': A), c ⊑ γ(a) ∧ c ↦ c' ∧ a ↦ a' ⇒  c' ⊑ γ(a')
%
The partial order for concrete domains is almost always the inclusion order on
𝒫(C), while the partial order for abstract domains is almost always a finite
element lattice.
%
A complete treatment of abstract interpretation is centered around constructing
a _galois connection_ between a concrete and abstract semantics. Establishing a
galois connection between C and A establishes both sides of the relationship
between C and A, whereas until now we've only had one direction.
Having both direction allows us to directly project concrete objects to
abstract objects, and provides a setting to discuss if γ is a "best"
abstraction function relating A to C.
%
A galois connection between two posets C and A is notated [C α⇄ γ A] and
consists of:
%
* [α : C → A] where α is monotonic
* [γ : A → C] where γ is monotonic
* [γ ∘ α] is expansive: ∀ (x : C), x ⊑ γ(α(x))
* [α ∘ γ] is contractive: ∀ (y : A), α(γ(y)) ⊑ y
%
The last two points can be succinctly stated as [α ∘ γ ⊑ id ⊑ γ ∘ α], using the
logical monotonicity relation [f ⊑ g ⇔  (x ⊑ y ⇒  f(x) ⊑ g(y))] for the
function space.
%
Equivalent to the above four rules is:
%
* x ⊑ γ(y) ⇔  α(x) ⊑ y
%
Example: given a galois connection (C α⇄ γ A), there exists a galois connection
(C → C) α⇄ γ (A → A) where:
%
α(f : C → C) = γ ∘ f ∘ α
γ(g : A → A) = α ∘ g ∘ γ
%
Using the constructed galois connection on the function space, we can more
succinctly state the correctness of an analysis as:
%
_↦_ ⊑ _↦ₐ_
%
Example: the language (𝒫(ℤ),+,*) forms a galois connection with the language
(𝒫({EVEN, ODD}),∧,∨) where:
%
α(zs : 𝒫(ℤ)) = ⋃ { { EVEN | ∃ z ∈ zs ∧ Even(z)}, { ODD | ∃ z ∈ zs ∧ Odd(z) } }
γ(ts : 𝒫({EVEN, ODD})) = ⋃ { { z ∈ ℤ | EVEN ∈ ts ∧ Even(z) }, { z ∈ ℤ | ODD ∈ ts ∧ Odd(z) } }
%
α and γ must have the expansive and contractive properties in order
for this to be considered a galois connection.  The expansive property
corresponds to the soundness of the abstraction.  Soundness ensures that if
you abstract and then project back to the concrete, you must cover the points
you started with.  The contractive property corresponds to the tightness of
the abstraction.  Tightness ensures that the abstraction is the most precise
abstraction possible.  For example, picking [α(zs) = {EVEN, ODD}] would be
sound, because projecting back to the concrete gives ℤ which tautologically
covers all integers, but this choice of α is clearly not tight.  Our choice
of α and γ in this example are indeed both sound and tight.
%
In this example, the concrete language is (ℤ,+,*) and the abstract language is
({EVEN,ODD},∧,∨)
%
Using the lifting of galois connections to the function space, we can/should
also justify _+_ ⊑ γ(_∧_) and _*_ ⊑ γ(_∨_)
%
% }}}

% Monadic AAM {{{

\section{Monadic AAM}
\label{section:MonadicAAM}

Monadic AI (Sergey et al. PLDI 2013) showed that computational modularity for
abstract interpreters can be achieved through a monadic abstraction.  We build
directly on this work at the level of intuition--monads serve as our
pivot-point for generalizing abstract control. However, in our pursuit of an
axis-independent correctness framework, our approach deviates greatly from this
work in the specific monadic interfaces used.

We use CPS to simplify presentation, although our implementation supports a
larger scheme-like language.

First we demonstrate a simple, fully instantiated instance of our framework
_without_ monads. This example is the instantiation of:

* A simple abstract domain which tracks control flow, booleans and integer
  constants.
* 0-cfa
* No abstract garbage collection or copied closurse (see mcfa)
* Fully flow- and path-sensitive, which is in some sense the most natural
  structural abstraction of the concrete state space (see AAM).

x,y,k : Var  ::= ..variables..
b,i,l : Lit  ::= ℤ ⋃ 𝔹
a     : Atom ::= x | l | op a | λ x . c | λ x k . c
op    : Op   ::= add1 | sub1 | gez
c     : Call ::= if(a){c}{c} | a(a) | a(a,a) | halt(a)

v  : ^Val^ ::= INT | l | <λ x . c> | <λ x k . c>

σ  : ^Store^ := Var ⇀  𝒫 (^Val^)

𝒜 [_](_) : (^Store^ × Atom) ⇀  𝒫 (^Val^)
𝒜 [σ](x)                      := σ(x)
𝒜 [σ](l)                      := { l }
𝒜 [σ](add1 a) | 𝒜 [ρ,σ] ⊑ INT := { INT }
𝒜 [σ](sub1 a) | 𝒜 [ρ,σ] ⊑ INT := { INT }
𝒜 [σ](gez a)  | 𝒜 [ρ,σ] ⊑ INT := { T , F }
𝒜 [σ](λ x k . c)              := { <λ x k . c> }

(this is often stated as a relation in the general setting of small-step
relational semantics.)

𝒞 <_,_> : (Call × ^Store^) ⇀  𝒫 (Call × ^Store^)
𝒞 <if(a){c₁}{c₂},σ> := { <c,σ> |  c  ∈ ⋃ [ { c1 } if T ∈ 𝒜 [σ](a), { c2 } if F ∈ 𝒜 [σ](a) ] }
𝒞 <a₁(a₂,a₃),σ>     := { <c,σ'> 
                       | <λ x k . c>  ∈ 𝒜 [σ](a₁)
                       |          v₂  ∈ 𝒜 [σ](a₂)
                       |          v₃  ∈ 𝒜 [σ](a₃)
                       |          σ' := σ ⊔ [x ↦ v₂] ⊔ [k ↦ v₃]
                       }

The complete analysis of a program c is defined as:

μ(Σ) . {<c,⊥>} ⊔ Σ ⊔ 𝒞*(Σ)

where

𝒞* : 𝒫 (Call × ^Store^) → 𝒫 (Call × ^Store)
𝒞*(Σ) = ⋃ [ 𝒞 <c,σ> | <c,σ> ∈ Σ ]

The monadic conversion of the above analysis is:

ℳ  : Set → Set
ℳ  a := StateT ^Store^ List a
ℳ  a := ^Store^ → 𝒫 (a × ^Store^)

𝒜 ₘ : Atom → ℳ (^Val^)
𝒜 ₘ(x) = do
  σ ← get-σ
  return σ(x)
𝒜 ₘ(l) = return {l}
𝒜 ₘ(add1 a) = do
  vD ← 𝒜 ₘ (a)
  return {INT | ∃ i ⊑ INT ∈ vD}
𝒜 ₘ(sub1 a) = do
  vD ← 𝒜 ₘ (a)
  return {INT | ∃ i ⊑ INT ∈ vD}
𝒜 ₘ(gez a) = do
  vD ← 𝒜 ₘ (a)
  return {T, F | ∃ i ⊑ INT ∈ vD}
𝒜 ₘ(λ x k . c) = return <λ x k . c>

𝒞ₘ : Call → ℳ (Call)
𝒞ₘ(if(a){c₁}{c₂}) := do
  vD ← 𝒜 ₘ (a)
  v ← lift𝒫 (vD)
  b ← coerce𝔹 (v)
  return { c₁ if b | c₂ if ¬b }
𝒞ₘ(a₁(a₂,a₃)) := do
  vD ← 𝒜 ₘ (a₁)
  v ← lift𝒫 (vD)
  <λ x k . c> <- coerceλ₂ v
  vD₂ ← 𝒜 ₘ (a₂)
  vD₃ ← 𝒜 ₘ (a₃)
  v₂ ← lift𝒫 vD₂
  v₃ ← lift𝒫 vD₃
  modify-σ (λ σ . σ ⊔ [x ↦ v₂] ⊔ [k ↦ v₃])
  return c

The monadic abstraction provides a nice way to simplify the implementation of
the analysis.  In particular, cases which do not modify the store or only
return one result--both of which are instances of having "no effect"--need not
mentioned all members of the state machine.  (Credit Sergey et. al.)

The complete analysis is defined as before where

𝒞ₘ* : 𝒫 (Call × ^Store^) → 𝒫 (Call × ^Store^)
𝒞ₘ*(Σ) = ⋃ [ (𝒞ₘ c)(σ) | <c,σ> ∈ Σ ]

The exact abstract interpreter written directly is recovered through the
definition of ℳ .  Our insight is twofold:

* ℳ  can be _axiomatized_, such that different definitions for ℳ  give rise to
  different control abstractions for the analysis.
* The axiomatized monadic abstraction provides for a modular proof framework
  for the correctness of a range of analyses.

% }}}

% Abstract Control {{{

\section{Abstract Control}
\label{section:AbstractControl}

The current instantiation of ℳ  yields a flow-sensitive path-sensitive
analysis.  Consider a reordering of the List and State monad transformers:

ℳ  : Set → Set
ℳ  a := SetT (State ^Store^) a
ℳ  a := ^Store^ → 𝒫 (a) × ^Store^

[Note: The definition of SetT we use is non-traditional.  However, our
definition for SetT is actually a monad, whereas traditional definitions are
not.  Our definition of SetT and proofs of monad laws are in the
correctness/proofs section.]

As before, we must convert between monadic actions and abstract state space
transitions to achieve an analysis:

𝒞ₘ* : 𝒫 (Call) × ^Store^ → 𝒫 (Call) × ^Store^
𝒞ₘ*(c𝒫,σ) = <{c' | c' ∈ π₁(𝒞ₘ c)(σ) | c ∈ c𝒫 }, ⋃ [ π₂(𝒞ₘ c)(σ) | c ∈ c𝒫 ]>

This instantiation of ℳ  and 𝒞ₘ* yields a flow-insensitive analysis.

For the same monad ℳ , we can change the definition of 𝒞ₘ* to achieve a
flow-sensitive path-insensitive analysis:

𝒞ₘ* : 𝒫 (Call × ^Store^) → 𝒫 (Call × ^Store^)
𝒞ₘ*(Σ) = { <c',π₂(𝒞 ₘ c)(σ)> | c' ∈ π₁(𝒞ₘ c)(σ) | c ∈ c𝒫 }

In the correctness framework, the flow-sensitive path-sensitive analysis and
flow-insensitive analyses are justified through isomorphisms between monadic
actions and abstract state space transitions.  The flow-sensitive
path-insensitive variant is recovered by weakening this isomorphism to a galois
connection.

% }}}

% Intensional Optimizations {{{

\section{Intensional Optimizations}
\label{section:IntensionalOptimizations}

We have now factored the abstract control properties of static analysis behind
a common interface.  Now we show how to implement two intentional
optimizations, abstract garbage collection and mcfa, in a completely general
setting.

The common interface for each variation in abstract control is a functor ℳ  ∈
Set → Set such that ℳ :

* Is a monad
  * return : ∀ (A : Set), A → ℳ  A
  * extend : ∀ (A β : Set), (A → ℳ  β) → (ℳ  A → ℳ  β)
* Is a state monad algebra on ^Store^
  * get : ℳ  ^Store^
  * put : ^Store^ → ℳ  1
* Is a monad plus
  * _+ₘ_ : ∀ (A : Set), ℳ  A → ℳ  A → ℳ  A

We abreviate the conjunction of the above properties holding on some ℳ  as
(Analysis ℳ ).

Abstract garbage collection can now be written once, generic to this interface:

gc : Call → ℳ  1
gc(c) = do
  σ ← get-σ
  touched₀ ← touched-𝒞(c)
  let touched := collect(touched-var)(touched₀)
  modify-σ (λ σ . ⋃ [ {x ↦ v} | σ(x) = v ∧ x ∈ touched ])

The mcfa optimization of copying rather than linking closures also enjoys a
fully generic implementation:

clo-copy : ([Var] × Call) → ℳ  Lam
clo-copy(xs,c) = do
  let ys := free-vars(c)
  σ ← get-σ
  vs ← map* σ(_) ys
  ρ ← { y ↦* v | (y,v) ∈ zip(ys, vs) }
  return <xs,c,ρ>

Using existing techniques, these optimizations must be proved correct for each
instantiation of abstract control.  Our generalization over abstract control
allows us to implement each of these optimizations once, and more importantly,
the proof about these optimizations also transfer to their instantiations for
free.

% }}}


% Correctness Framework {{{

\section{Correctness}
\label{section:Correctness}

The key advantage to our formulation of monadic abstract interpretation--both
over non-monadic approaches like AAM, and the monadic approach described by
Sergey et al. (PLDI 2013)--is that the proofs of correctness for the
constructed analyses can be derived automatically alongside the computational
artifact.

Our monadic abstraction is a proper abstraction in that it mediates between two
parties--the framework and the analysis designer--and gives guarantees to both
sides conditioned on assumptions being met.

We show that language-independent monads exist for instantiating a semantics to
flow-sensitive and path-sensitive analyses.

In the core analysis framework we prove:

* SetT and StateT are monad transformers and compose in both directions.
* Monadic actions (A → ℳ  β) in SetT and StateT form galois connections with
  abstract state machine transitions (𝒮𝒮ₘ A → 𝒮𝒮ₘ β).
* One composition of SetT and StateT is in (⊑) relation with the other.
* For (SetT ∘ StateT), one choice of state machine 𝒮𝒮ₘ is in (⊑) relation with
  the other.

The analysis designer must then prove:

* The semantic step function 𝒞ₘ (which calls 𝒜 ₘ) is monotonic, including the
  choice of 𝓂 .
* The semantics as written, including intensional optimizations, are correct.

Given proofs from both sides of the interface we can conclude:

flow-insensitive 0-cfa 
 ⊑ 
flow-sensitive path-insensitive 0-cfa 
 ⊑ 
flow-sensitive 0-cfa
 ⊑ 
concrete cfa

This proof technique is maximally modular, and greatly reduces the proof burden
on the analysis designer as new languages and analyses are developed.

When adding the rest of the analysis tuning knobs like abstract domain, context
sensitivity, intensional optimizations and object sensitivity, the monotonicity
of 𝒞ₘ must be re-established.  However, like for abstract control, galois
connections can be built between these choices independent of their
instantiation in 𝒞ₘ. Our implementation supports each of these axis in full.

% }}}

% Proofs and Definitions {{{

\section{Proofs and Definitions}
\label{section:Proofs+Definitions}

% Setup {{{

\subsection{Setup}
\label{section:Proofs+Definitions:Setup}

PartialOrder(A : Set) :=
  operators
    _⊑_ : A → A → Prop
  laws
    reflexivity : x ⊑ x
    antisymmetry : x ⊑ y ⇒  y ⊑ x ⇒  x = y
    transitivity : x ⊑ y ⇒  y ⊑ z ⇒  x ⊑ z

Monotonic(𝑓 : A → β) := 
  require
    PartialOrder(A)
    PartialOrder(β)
    ∀ (x y : A), x ⊑ y ⇒  f(x) ⊑ f(y)

(C : Set) α⇄ γ (A : Set) :=
  require
    PartialOrder(C)
    PartialOrder(A)
  operators
    α : C → A
    γ : A → C
  laws
    Monotonic(α)
    Monotonic(γ)
    α ∘ γ ⊑ id ⊑ γ ∘ α

JoinSemilattice(A : Set) :=
  require
    PartialOrder(A)
  operators
    ⊥ : A
    _⊔_ : A → A → A
  laws
    left-unit : ⊥ ⊔ x = x
    right-unit : x ⊔ ⊥ = x
    associativity : x ⊔ (y ⊔ z) = (x ⊔ y) ⊔ z
    commutativity : x ⊔ y = y ⊔ x

Monad(𝓂  : Set → Set) :=
  operators
    return : ∀ (A : Set) → A → 𝓂 (A)
    extend : ∀ (A β : Set) → (A → 𝓂 (β)) → (𝓂 (A) → 𝓂 (β))
  laws
    left-unit : extend(return) = id
    right-unit : extend(k)(return(A)) = k(A)
    associativity : extend(k₂)(extend(k₁)(aM)) = extend(λ x → extend(k₂)(k₁(x)))(A)

Transformer(𝒯 : (Set → Set) → (Set → Set)) :=
  require
    ∀ (𝓂  : Set), Monad(𝓂 ) ⇒  Monad(𝒯(𝓂 ))
  operators
    lift : ∀ (𝓂  : Set), 𝓂 (A) → 𝒯(𝓂 )(A)
  laws
    unit : lift(return) = return

MonadState(𝓈 : Set)(𝓂  : Set → Set) :=
  operators
    get : 𝓂  𝓈
    put : 𝓈 → 𝓂  1

MonadPlus(𝓂  : Set → Set) := 
  require
    ∀ (A : Set), JoinSemilattice(𝓂 (A))
  laws
    left-zero : extend(k)(⊥) = ⊥
    right-zero : extend(const(⊥))(aM) = ⊥
    distributivity : extend(k)(aM₁ ⊔ aM₂) = extend(k)(aM₁) ⊔ extend(k)(aM₂)

Functorial(ℭ : Set → Prop)(𝓂  : Set → Set) := 
  require
    ∀ (A : Set), ℭ (A) → ℭ (𝓂 (A))

Pointed(𝓂  : Set → Set) :=
  unit : ∀ (A : Set) → A → 𝓂 (A)

MonadSmallStep(𝓂  : Set → Set)(𝒮𝒮 : Set → Set) := 
  require
    ∀ (A β : Set), (A → 𝓂 (β)) α⇄ γ (𝒮𝒮(A) → 𝒮𝒮(β))

% }}}

% SetT {{{

\subsection{SetT}
\label{section:Proofs+Definitions:SetT}

SetT : (Set → Set) → (Set → Set)
SetT(𝓂 )(A) := 𝓂 (𝒫 (A))

-----------------
Transformer(SetT)

lift : ∀ (𝓂  : Set → Set) (A : Set), 𝓂 (A) → SetT(𝓂 )(A)
lift := map (λ x → { x })

Monad(𝓂 ) ∧ Functorial(JoinSemilattice)(𝓂 )
-------------------------------------------
Monad(SetT(𝓂 ))

return : ∀ (A : Set), A → SetT 𝓂  A
return := lift ∘ returnₘ

extend : ∀ (A β : Set), (A → SetT(𝓂 )(β)) → (SetT(𝓂 )(A) → SetT(𝓂 )(β))
extend(k) := joinₘ ∘ mapₘ (joins ∘ mapₚ k)

Monad(𝓂 ) ∧ Functorial(JoinSemilattice)(𝓂 )
-------------------------------------------
MonadPlus(SetT(𝓂 ))

m0 : ∀ (A : Set), SetT 𝓂  A
m0 = ⊥ ₘ

_<+>_ : ∀ (A : Set) → SetT 𝓂  A → SetT 𝓂  A → SetT 𝓂  A
_<+>_ := _⊔_

Monad(𝓂 ) ∧ MonadState(𝓈)(𝓂 )
-----------------------------
MonadState(𝓈)(𝓂 )

get : SetT 𝓂  𝓈
get = lift getₘ

put : 𝓈 → SetT 𝓂  1
put = lift ∘ putₘ

MonadSmallStep(𝓂 )(𝒮𝒮) ∧ Functorial(JoinSemilattice)(𝓂 ) ∧ Functor(𝒮𝒮)
-------------------------------
MonadSmallStep(SetT 𝓂 )(𝒮𝒮 ∘ 𝒫)

α : ∀ (A β : Set), (A → SetT(𝓂 )(β)) → (𝒮𝒮(𝒫 (A)) → 𝒮𝒮(𝒫 (β)))
α(f) = αₘ(joinsₘ ∘ mapₚ(f))

γ : ∀ (A β : Set), (𝒮𝒮(𝒫 (A)) → 𝒮𝒮(𝒫 (β))) → (A → SetT(𝓂 )(β))
γ(f) = γₘ(f ∘ mapₛₛ(unitₚ))

MonadSmallStep(𝓂 )(𝒮𝒮) ∧ (𝒫 ∘ 𝒮𝒮) α⇄ γ (𝒮𝒮 ∘ 𝒫 )
-------------------------------------
MonadSmallStep(SetT 𝓂 )(𝒫 ∘ 𝒮)

α : ∀ (A β : Set), (A → SetT(𝓂 )(β)) → (𝒫 (𝒮𝒮(A) → 𝒫 (𝒮𝒮(β))))
α(f) = extendₚ(α ∘ αₘ(f))

γ : ∀ (A β : Set), (𝒫 (𝒮𝒮(A) → 𝒫 (𝒮𝒮(β)))) → (A → SetT(𝓂 )(β))
γ(f) = extendₚ(γ ∘ γₘ(f))

% }}}

% StateT {{{

\subsection{StateT}
\label{section:Proofs+Definitions:StateT}

StateT : Set → (Set → Set) → (Set → Set)
StateT(𝓈)(𝓂 )(A) := 𝓈 → 𝓂  (A × 𝓈)

Transformer(StateT(𝓈))
----------------------

lift : ∀ (𝓂  : Set → Set) (A : Set), 𝓂  a → StateT(𝓈)(𝓂 )(A)
lift aM := λ 𝓈 → mapₘ (,𝓈) aM

Monad(𝓂 )
--------------------
Monad(StateT(𝓈)(𝓂 ))

return : ∀ (A : Set), A → StateT(𝓈)(𝓂 )(A)
return := lift ∘ returnₘ

extend : ∀ (A β : Set), (A → StateT(𝓈)(𝓂 )(β)) → (StateT(𝓈)(𝓂 )(A) → StateT(𝓈)(𝓂 )(β))
extend(k)(aM) := λ 𝓈 → doₘ
  (a,𝓈') ← aM(𝓈)
  k(a)(𝓈')

JoinSemilattice(𝓈) ∧ Monad(𝓂 )
------------------------------------------
Functorial(JoinSemilattice)(StateT(𝓈)(𝓂 ))

bot : StateT(𝓈)(𝓂 )(A)
bot = λ _ → (⊥ ₛ, ⊥ ₐ)

_⊔_ : StateT(𝓈)(𝓂 )(A) → StateT(𝓈)(𝓂 )(A) → StateT(𝓈)(𝓂 )(A)
aM₁ ⊔ aM₂ := λ 𝓈 → doₘ
  (a₁,𝓈₁) ← aM₁
  (a₂,𝓈₂) ← aM₂
  return (a₁ ⊔ a₂, 𝓈₁ ⊔ 𝓈₂)

MonadPlus(𝓂 )
------------------------
MonadPlus(StateT(𝓈)(𝓂 ))

⊥ : ∀ (A : Set), StateT(𝓈)(𝓂 )(A)
⊥ := lift ⊥ ₘ

_⊔_ : ∀ (A : Set), StateT(𝓈)(𝓂 )(A)
aM₁ ⊔ aM₂ := λ 𝓈 → aM₁(𝓈) <+> aM₂(𝓈)

Monad(𝓂 )
----------------------------
MonadState(𝓈)(StateT(𝓈)(𝓂 ))

get : StateT 𝓈 𝓂  𝓈
get = λ 𝓈 → return (𝓈,𝓈)

put : 𝓈 → StateT 𝓈 𝓂  1
put 𝓈 = λ _ → return (∙,𝓈)

MonadSmallStep(𝓂 )(𝒮𝒮)
--------------------------------------
MonadSmallStep(StateT(𝓈)(𝓂 ))(𝒮𝒮(_ × 𝓈))

α : ∀ (A B : Set), (A → StateT(𝓈)(𝓂 )(B)) → (𝒮𝒮(A × 𝓈) → 𝒮𝒮(B × 𝓈))
α(f) = αₘ (λ (a,𝓈) → f(a)(𝓈))

γ : ∀ (A B : Set), (𝒮𝒮(A × 𝓈) → 𝒮𝒮(B × 𝓈)) → (A → StateT(𝓈)(𝓂 )(B))
γ(f) = λ a 𝓈 → γₘ (f)(a,s)

% }}}

% Propositions {{{

\section{Propositions}
\lable{section:Proofs+Definitions:Propositions}

* SetT and StateT are Galois Functors.
* 𝒜 ₘ and 𝒞ₘ are monotonic (including in 𝓂 ).

% }}}

% }}}

\bibliography{davdar}{}
\bibliographystyle{plain}

\end{document}
