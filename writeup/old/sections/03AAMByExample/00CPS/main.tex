The state space for CPS syntactically separates \h|Call| expressions from \h|Atom|
expressions.
%
\haskell{sections/03AAMByExample/00CPS/00Language.hs}

%--% 

\h|Call| expressions are separated from \h|Atom| expressions because they may
not terminate, and therefore cannot be given a striaghtforward denotation.
%
Each syntactic form in \h|Call| is designed to be recursive in only one position,
eliminating the need for a call stack when we design the operational semantics.

%--%

\h|Atom| expressions are exactly those for which we can compute a denotation.
%
The \h|Lam| syntactic form may look troubling because it is mutually recursive
with \h|Call|, but the denotation of \h|Lam xs c| will be a closure which delays
the evaluation of \p|c| until it is applied in a \h|Call| rule.

%--%

Possibly non-standard additions to our CPS language are conditional branching
(\h|IfC|); two literal types, integer and boolean; and primitive operations
\h|Add1|, \h|Sub1| and \h|IsNonNeg|.
%
We include these to bring CPS closer to a language one might use in practice,
and to examine the design space of their abstract semantics.


