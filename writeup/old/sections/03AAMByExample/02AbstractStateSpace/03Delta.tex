Now we have a \textit{nondterministic} semantics paramterized by abstract
addresses--but we are not yet finished.
%
The state space is still infinite due to the unboundedness of integer literal
values.
%
(Even if we took the semantics of finite machine integers, $2^{32}$ is still much
too large a state space to be exploring in a practical analysis.)
%
To curb this we introduce yet another axis of parameterization with the intent
of recovering multiple analyses from the choice later.
%
In the Haskell implementation we represent this axis with the type class
\h|Delta| and associated type \h|Val|.
%
\begin{figure}[H]
\haskell{sections/03AAMByExample/02AbstractStateSpace/03Delta.hs}
\caption*{(Final) Delta Abstraction}
\end{figure}
\noindent
%
Different analyses may want to make different choices in their value
representations, so we don't commit to a particular \h|Val| type, rather we
require that \textit{some} \h|Val| type exist with a particular interface.
%
All that matters is that we have \textit{introduction} forms for literals and
closures, a \textit{denotation function} for primitive operations, and
\textit{elimination} forms for booleans and closures.
%
Eliminators are only required for booleans and closures because those are the
only values scrutenized in the control flow of the semantics.
