Abstract garbage collection has become standard in structural abstract
interpretation and a staple in the AAM methodology.
%
The idea is analogous to garbage collection for memory safe languages.
%
Even though we must accept that abstract addresses can be reused during
execution, it would be nice if we could completely replace the contents of an
abstract location if we knew no future state would need the old value.
%
This concept--of future states depending on values in the store--can be
approximated by the \textit{reachability} of those values in the current state.
%
Abstract garbage collection is the process where unreachable allocations in the
store are removed so that future allocations don't require joins in the store.

%--%

To perform abstract garbage collection we must examine all addresses in the
store and eliminate those which cannot be reached from a \textit{root} in the
state space.
%
The roots $\mathcal{T}(c,e)$ in the state space are the locations bound to free
variables in $c$:
%
\begin{equation*}
\mathcal{T}(c,e) = \{e(v) : v \in free(c)\}
\end{equation*}
%
The reachable addresses $\mathcal{R}(c,e,s,t)$ are those which are reachable
from the roots, which is the least solution to the equation:
%
\begin{equation*}
\mathcal{R}(c,e,s,t) = \mathcal{T}(c,e) 
                  \cup \{ \ell' 
                        : \ell \in \mathcal{R}(c,e,s,t)
                        , \ell \leadsto_{\langle e,s \rangle} \ell'
                       \}
\end{equation*}
where $\leadsto_{\langle e,s \rangle}$ is the single-step points-to relation
in environment $e$ and store $s$.
%
The set of addresses to eliminate is the complement of $\mathcal{R}(c,e,s,t)$
