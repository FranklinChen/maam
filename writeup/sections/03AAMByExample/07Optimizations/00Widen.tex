The state space we arrived at from our methodological abstraction of the
concrete state space in section ~\ref{section:AAMByExample:AbstractStateSpace}
was 
%
\begin{center}
\h|StateSpace d aam = Set (Call, Env aam, Store d aam, Time aam)|
\end{center}
%
This state space will lead to a fully flow sensitive analysis, and consequently
an exponential runtime.
%
The fix is well known in the literature as \textit{heap widening}, where the
desired state space is 
%
\begin{center}
\h|StateSpace d aam = (Set (Call, Env aam, Time aam), Store d aam)|.
\end{center}
%
When widening the heap, each store in the set of states is joined together to
form a single universally approximating store.
%
This results in a less precise but vastly more efficient abstraction.

%--%

Introducing heap widening into the analysis can be accomplished in two ways:
rewriting all the semantics for the new state space, or by introducing a
post-facto widening operation that operates on the original state space.
%
We demonstrate the latter approach, which can be understood as a Galois
connection between the inefficient and efficient state spaces.
%
\begin{center}
\h|Set (Call, Env, Store, Time)| 
\galois{\alpha}{\gamma} 
\h|(Set (Call, Env, Time), Store)|
\end{center}
%
We show only the $\alpha$ portion of the Galois connection and call it
\h|widen|.
%
\haskell{sections/03AAMByExample/07Optimizations/00Widen.hs}
