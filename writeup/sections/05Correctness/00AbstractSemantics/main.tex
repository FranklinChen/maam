In order to obtain Galois connections for our abstract interpreters, it must
first be established that \h|step| is monotonic.
%
Because \h|step| is highly parameterized, we must place monotonicity conditions
on these parameters in order to arrive at a final monotonicity argument for
\h|step|.

%--%

We remind the reader that the parameters to the \h|step| function are:
%
\begin{itemize}
\item The \h|Delta d| parameter, which specifies the value type and semantics of
      primitive operations.
\item The \h|AAM aam| parameter, which specifies abstract time and address types
      and the semantics of allocation.
\item The \h|Monad m| parameter, which specifies the computational properties
      of the abstraction, like nondeterminism or heap widening.
\end{itemize}
%
To justify that \h|step| is monotonic, we require \textit{all} base types in
the parameters (like \h|Val d|, \h|Time aam| and \h|Addr aam|) to come equipped
with partial orders, and \textit{all} functions provided by the the parameters
to be monotonic.
%
We must also clarify that we are using the discrete order for all base types
(\h|Bool|, \h|Name|, etc.), and the standard subset inclusion ordering for
\h|Set| and \h|Point| types.

%--%

\newtheorem{atom-call-step-monotonic}{Lemma}
\begin{atom-call-step-monotonic}
  \h|atom|, \h|call| and \h|step| are monotonic.
\end{atom-call-step-monotonic}
\begin{proof}
  Follows trivially from the assumed monotonicity of \h|Delta|, \h|AAM| and
  \h|Monad| operations.
\end{proof}
